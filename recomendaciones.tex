Es importante que en la escuela de ingeniería eléctrica de la UCV se profundice en el uso de técnicas de visión para el posicionamiento y estimación de distancias, puesto que a pesar de que sus resultados no son tan precisos como los de un sensor LIDAR, estas pueden generar implementaciones en campos tan extensos como lo son el de la robótica o el de visión por computador, que logran ser económicamente más rentables y en conjunto con otro tipo de sensores mediante fusión sensorial pueden alcanzar resultados notables en dichos campos.
\\
\\
Por este motivo se recomienda colaborar en el proyecto de código abierto py2vision para que se convierta en una herramienta de visión estéreo que cuente con una amplia variedad de técnicas de medición y diversidad de estrategias de reconocimiento.
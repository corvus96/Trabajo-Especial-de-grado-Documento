Se recomienda que se profundice en el uso de técnicas de visión para el posicionamiento y estimación de distancias, que utilicen \textit{deep learning} en el cálculo de mapas de disparidad densos, ya que esto elimina la necesidad de ajustar los parámetros del método de correspondencia y simplifica la etapa de calibración, lo cual puede llevar un tiempo considerable hasta alcanzar resultados óptimos. También se recomienda comparar el uso de estas estrategias que empleen únicamente deep learning con metodologías híbridas como la implementada en este trabajo de grado. El estudio de las técnicas estéreo posee la gran ventaja respecto a un sensor LIDAR de lograr ser económicamente más rentables.
\\
\\
Por este motivo se recomienda continuar investigaciones del ámbito estéreo por medio del proyecto py2vision, para que se convierta en una herramienta de visión estéreo, que cuente con una amplia variedad de técnicas de medición y diversidad de estrategias de reconocimiento, al alcance de los integrantes de la Escuela de Ingeniería Eléctrica de la UCV.
Se entiende por posicionamiento en el campo de visión por computador (VC) a todas aquellas técnicas que permiten extraer las coordenadas espaciales de un objeto a partir de su representación en 2 dimensiones. Entre las técnicas empleadas para el posicionamiento, aquellas que utilizan visión estéreo son ampliamente utilizadas en el campo de la robótica y los vehículos autónomos. Este hecho no es de extrañar, debido a que los inicios de esta tecnología datan de 1838 cuando Sir Charles Wheatstone publico un articulo en el que describía la visión estereoscópica \cite{Wheatstone1837}, en dicho articulo explica el como esta tecnología se basaba en la visión humana y en la separación de, aproximadamente, 65 mm que existe entre nuestros ojos. Estos reciben cada uno una imagen diferente que el cerebro une creando el efecto de tridimensionalidad. Por supuesto no tardo mucho hasta que decidieron aplicar esta teoría en el invento de moda de la época, las cámaras, y hoy en día el camino que se ha recorrido para permitir que un computador pueda interpretar coordenadas espaciales de una imagen nos ha permitido emplear las cámaras como sensores que interpretan el entorno que les rodea. 

En robótica se han desarrollado diversas metodologías de control basadas en sensores que dotan al robot de sentidos para percibir el mundo que le rodea. Algunas de las técnicas más usadas en el entorno industrial dependen de la odometría, que no es mas que el estudio de  la estimación de la posición de vehículos con ruedas durante la navegación. Para realizar esta estimación se usa información sobre la rotación de las ruedas mediante encoders con el fin de estimar cambios en la posición a lo largo del tiempo. Por otro lado, las técnicas de visión por computador en las décadas recientes, se han vuelto cada vez mas importantes debido al incremento en la capacidad de procesamiento y almacenamiento de los dispositivos electrónicos, además de la creciente necesidad realizar tareas más complejas con los autómatas, son flexibles a tal punto de que pueden ser implementadas con una o varias cámaras, las imágenes captadas son preprocesadas mediante filtros y luego con algoritmos, es posible determinar que objetos se encuentran en la imagen. Si se emplean metodologías basadas en visión estéreo, cámaras RGBD o láser, el robot gana la capacidad de localizarse en el entorno de trabajo o localizar objetos respecto al mismo.

En la actualidad las técnicas mas avanzadas de visión por computador (VC) utilizan las abstracciones conocidas como redes neuronales artificiales (RNA), las cuales suelen ser programadas en frameworks OpenSource basados en python, orientados al aprendizaje automático (AA). Algunos de los mas populares son Tensorflow o Keras por su compatibilidad con dispositivos embebidos  e incluso OpenCV para el preprocesamiento de las imágenes. Estas redes son capaces de aprender patrones en conjuntos de datos que se le suministra y dependiendo de como se le entrene pueden generalizar para datos que no se encuentren en el conjunto de datos de entrenamiento. En el campo de VC se traduce en detectar objetos para obtener la localización del mismo en una imagen o clasificar el tipo de objeto.

En la industria Venezolana no son muy comunes desarrollos en el campo de la localización de objetos en una escena, por lo que las técnicas de visión para el control no son muy comunes, de modo que las estrategias de control de robots suelen basarse en odometría o fusión de sensores, sumados a un conjunto de restricciones e instrucciones que controlen el movimiento de los autómatas. Por este motivo se propone desarrollar un paquete en Python que permita a programadores e ingenieros implementar sistemas de control cuya data sean imágenes provenientes de escenas estereoscópicas, los cuales podrán ser implementados en robots de uso domestico e industrial.
\renewcommand{\baselinestretch}{1.0}% Espaciado entre linea
\begin{titlepage}

\setlength{\unitlength}{1cm}%
\begin{picture}(5,5)(-5,0)
\put(-6,3){{
\begin{minipage}[h]{2cm}
%\includegraphics[width=2cm]{ucv.eps}
%\includegraphics[width=2cm]{newton.eps}
\end{minipage}}
}%
\put(-4,4){{
\begin{minipage}[h]{11cm}
\begin{center}
\begin{large}
\textbf{TRABAJO ESPECIAL DE GRADO}

%Facultad de Ingeniería

%Escuela de Ingeniería Eléctrica

\end{large}
\end{center}
\end{minipage}}
}%
\put(8,3){{
\begin{minipage}[h]{2cm}
%\includegraphics[width=2cm]{fi.eps}
%\includegraphics[width=2cm]{lagrange.eps}
\end{minipage}}
}%
\put(1,-12){{
\begin{minipage}[h]{8cm}
\begin{flushright}
\renewcommand{\baselinestretch}{1.0}% Espaciado entre linea
\begin{spacing}{1}
    Presentado ante la ilustre\\
Universidad Central de Venezuela\\
por el Br. Guillermo Raven\\
para optar al título de \\
Ingeniero Electricista.
\end{spacing}
\end{flushright}

\end{minipage}}
}%

\put(-1,-16){{
\begin{minipage}[h]{8cm}
Caracas, noviembre de 2020
\end{minipage}}
}%

\end{picture}
\begin{center}
\vspace{2.1cm}%
\begin{large}
\textbf{DESARROLLO DE UN PAQUETE EN PYTHON PARA EL POSICIONAMIENTO DE OBJETOS EN UNA ESCENA MEDIANTE VISIÓN ESTEREOSCÓPICA Y TÉCNICAS DE RECONOCIMIENTO BASADAS EN APRENDIZAJE AUTOMÁTICO
 }
\end{large}
\end{center}
\end{titlepage}

%%%%%%%%%%%%%%%%%%%%%%%%%%%%%%%%% Anteportada %%%%%%%%%%%%%%%%%%%%%%%%%%%%%%%%%%%%%%%%%
\newpage


\begin{titlepage}

\setlength{\unitlength}{1cm}%
\begin{picture}(5,5)(-5,0)
\put(-6,3){{
\begin{minipage}[h]{2cm}
%\includegraphics[width=2cm]{ucv.eps}
%\includegraphics[width=2cm]{newton.eps}
\end{minipage}}
}%
\put(-4,4){{
\begin{minipage}[h]{11cm}
\begin{center}
\begin{large}
\textbf{TRABAJO ESPECIAL DE GRADO}

%Facultad de Ingeniería

%Escuela de Ingeniería Eléctrica

\end{large}
\end{center}
\end{minipage}}
}%
\put(8,3){{
\begin{minipage}[h]{2cm}
%\includegraphics[width=2cm]{fi.eps}
%\includegraphics[width=2cm]{lagrange.eps}
\end{minipage}}
}%
\put(2,-12){{
\begin{minipage}[h]{8cm}
\begin{flushright}
\begin{spacing}{1}
    Presentado ante la ilustre\\
Universidad Central de Venezuela\\
por el Br. Guillermo Jose Raven Lusinche\\
para optar al título \\
de Ingeniero Electricista.
\end{spacing}
\end{flushright}

\end{minipage}}
}%

\put(-5.8,-8.5){{
\begin{minipage}[h]{11cm}
PROFESORA GUÍA: Profesora Tamara Pérez\\
TUTOR INDUSTRIAL: Ingeniero Carlos Rodríguez
\end{minipage}}
}%

\put(-1,-16){{
\begin{minipage}[h]{8cm}
Caracas, noviembre de 2020
\end{minipage}}
}%

\end{picture}
\begin{center}
\vspace{2.1cm}%
\begin{large}
\textbf{DESARROLLO DE UN PAQUETE EN PYTHON PARA EL POSICIONAMIENTO DE OBJETOS EN UNA ESCENA MEDIANTE VISIÓN ESTEREOSCÓPICA Y TÉCNICAS DE RECONOCIMIENTO BASADAS EN APRENDIZAJE AUTOMÁTICO
 }
\end{large}
\end{center}
\end{titlepage}
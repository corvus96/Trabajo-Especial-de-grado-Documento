Encontrar la posición de objetos a través de un sistema estéreo que utiliza la disparidad como regla de medición y la detección de objetos como forma de reconocimiento, es un proceso que puede ser resumido en: identificar los parámetros de las cámaras, pre-procesar las imágenes captadas, hallar el mapa de disparidad, predecir los bounding boxes, segmentar el área de dichos recuadros y finalmente calcular las coordenadas de los píxeles segmentados mediante triangulación fundamentada en los mapas de disparidad y las cualidades físicas del sistema.
\\
Aunque el proceso parezca complejo, este se simplifica con el paquete Py2vision, de tal forma que el usuario solo tenga que concentrar sus esfuerzos en ajustar los parámetros de los mapas de disparidad a través de una interfaz y proporcionar el conjunto de datos necesarios para el entrenamiento de una red junto a sus anotaciones en un formato compatible. De igual forma el paquete está diseñado para que pueda ser implementado dicho sistema en pocas líneas de código o probando diferentes configuraciones de red, con un algoritmo de correspondencia distinto a SGBM. Al mismo tiempo tiene la capacidad de hallar un mapa de profundidad que pueda ser usado para el modelado 3D, hallar únicamente el mapa de disparidad o calibrar una sola cámara, es decir; el paquete otorga flexibilidad de implementación para cualquiera de sus principales cualidades.
\\
\\
A su vez cuenta con un conjunto de lecciones para aprender a utilizar sus funcionalidades y una documentación completa de cada clase, método y función. No obstante, lo más importante es que elimina la complejidad de implementar un sistema estéreo o de reconocimiento, el cual puede ser utilizado en robótica y aplicaciones domésticas de visión por computador.
\\ 
En definitiva los sistemas aplicables con py2vision tendrán un grado de precisión que dependerá del conjunto de datos de entrenamiento, así como de la selección de sus hiperparámetros y la calidad del mapa de disparidad obtenido. Mientras que su velocidad, está relacionada con la potencia gráfica de la GPU, ya que su límite, es el propio del modelo utilizado en el reconocimiento, en este caso fue YOLO V3.
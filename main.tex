\documentclass[letterpaper,titlepage,12pt,oneside,spanish,final]{report_eie}

%\documentclass[letterpaper,titlepage,12pt,twoside,openright,spanish,final]{report_eie}

%%%%%%%%%%%%%%%%%%%%%%%%%%%%%%%%%%%%%%%%%%%%%%%%%%%%%%%%%%%%%%%%%%%%%%%%
\usepackage[spanish]{babel}
\usepackage[utf8]{inputenc}
\usepackage[T1]{fontenc}  %Estilo de fuente time new roman
\usepackage{amssymb}
\usepackage{amsfonts}
\usepackage{amsmath}
\usepackage{latexsym}
\usepackage[letterpaper]{geometry}

\usepackage{float}
\usepackage{makeidx}
\usepackage{color}
\usepackage{tocbibind}
\usepackage{acronym}
\usepackage{epsfig}
\usepackage{graphicx}
\usepackage{setspace}
\usepackage{multicol}
\usepackage{longtable}
%\usepackage{doublespace}
\usepackage[document]{ragged2e}
\usepackage{fancyhdr}
%\usepackage{fancyheadings}
\usepackage{indentfirst}
\usepackage{booktabs}
\usepackage{minted} % Formato de codigo en python
\usepackage{listings}
\usepackage{caption}
\usepackage{subcaption}
%========= Define el estilo de referencias ===============
%\usepackage[round,authoryear]{natbib}%\usepackage[square,numbers]{natbib}%
%\usepackage[comma,authoryear]{natbib} esto está abajo

%========= Define el estilo de referencias APA ===============
\usepackage[
backend=biber,
style=ieee,
citestyle=numeric
]{biblatex}

\addbibresource{referencias.bib} %Imports bibliography file
\usepackage{csquotes}
\usepackage[compact]{titlesec} %modificar espaciado


\usepackage{url}
\usepackage{hyperref}
%\usepackage[dvips,colorlinks=true,urlcolor=red,citecolor=black,anchorcolor=black,linkcolor=black]{hyperref}
%%%%%%%%%%%%%%%%%%%%%%%%%%%%%%%%%%%%%%%%%%%%%%%%%%%%%%%%%%%%%%%%%%
%            Definición del Documento PDF, (PDFLaTeX)            %
%%%%%%%%%%%%%%%%%%%%%%%%%%%%%%%%%%%%%%%%%%%%%%%%%%%%%%%%%%%%%%%%%%

\hypersetup{pdfauthor=Guillermo Raven}

\hypersetup{pdftitle=Trabajo Especial de Grado}%

\hypersetup{pdfkeywords=Robot manipulador}

\pdfstringdef{\Produce}{Escuela de Ingeniería Eléctrica, Facultad de Ingeniería, UCV}%

\pdfstringdef{\area}{Área del trabajo}

\hypersetup{pdfproducer=\Produce}

\hypersetup{pdfsubject=\area}

\hypersetup{bookmarksnumbered=true}

%%%%%%%%%%%%%%%%%%%%%%%%%%%%%%%%%%%%%%%%%%%%%%%%%%%%%%%%%%%%%%%%%%
%\setcounter{MaxMatrixCols}{10}


%===================== Re-definición de Ambientes =================
\newtheorem{theorem}{Teorema}
\newtheorem{acknowledgement}[theorem]{Acknowledgement}
\newtheorem{algoritmo}[theorem]{Algorithm}
\newtheorem{supuestos}[theorem]{Supuestos}
\newtheorem{hipotesis}[theorem]{Hipótesis}
\newtheorem{axiom}[theorem]{Axiom}
\newtheorem{case}[theorem]{Case}
\newtheorem{claim}[theorem]{Claim}
\newtheorem{conclusion}[theorem]{Conclusión}
\newtheorem{condition}{Condición}
\newtheorem{conjecture}{Conjecture}
\newtheorem{corollary}{Corollary}
\newtheorem{criterion}{Criterion}
\newtheorem{definition}{Definición}  %{Definition}
\newtheorem{example}[theorem]{Ejemplo}%{Example}
\newtheorem{exercise}[theorem]{Exercise}
\newtheorem{lemma}{Lemma}
\newtheorem{notation}[theorem]{Notation}
\newtheorem{problem}{Problem}
\newtheorem{property}{Property}
\newtheorem{proposition}{Proposition}
\newtheorem{remark}[theorem]{Remark}
\newtheorem{solution}{Solution}
\newtheorem{summary}[theorem]{Summary}
\newenvironment{proof}[1][Proof]{\noindent\textbf{#1.} }{\ \rule{0.5em}{0.5em}}%

\numberwithin{equation}{chapter}%
\numberwithin{figure}{chapter}%
\numberwithin{table}{chapter}%
\numberwithin{definition}{chapter}%
\numberwithin{lemma}{chapter}%
\numberwithin{theorem}{chapter}%
\numberwithin{corollary}{chapter}%
\numberwithin{condition}{chapter}%
\numberwithin{criterion}{chapter}%
 \numberwithin{problem}{chapter}%
\numberwithin{property}{chapter}%
\numberwithin{proposition}{chapter}%
\numberwithin{solution}{chapter}%
\numberwithin{conjecture}{chapter}%

%==================== Separación en sílabas ========================
\hyphenpenalty=6800%
\input{silabear.tex}

%==================== Diseño de Página =============================
%\pagestyle{headings}
%\setlength{\headheight}{0.2cm}
\setlength{\textwidth}{14.52cm}%
%\pagestyle{fancy}
%\renewcommand{\sectionmark}[1]{\markright{\thesection\ #1}}
%\rhead[\fancyplain{}{\bfseries\thepage}]{\fancyplain{}{\bfseries\rightmark}}%\thepage
%\lhead[\fancyplain{}{\bfseries\leftmark}]{\fancyplain{}{\bfseries}} \cfoot{}%

%\fancyhead[R]{}


\rfoot[\fancyplain{}{\textit{E. Brea}}] {\fancyplain{}{}}
\lfoot[\fancyplain{}{}] {\fancyplain{}{\textit{}}}    %%%%%%%%%%%%%%%%%%% OJO ACA %%%%%%%%%%
\cfoot[\fancyplain{}{}] {\fancyplain{}{\bfseries\thepage}}
%\setlength{\footrulewidth}{0.0pt}%
%\setlength{\headrulewidth}{0.1pt}%

%===================================================================



%================== Diseño de Párrafo y delimitador ================
\renewcommand{\baselinestretch}{1.5}% Espaciado entre linea
\geometry{left=4cm,right=3cm,top=3cm,bottom=3cm}
\frenchspacing %
%\raggedright % Sólo para justificar el texto a la izquierda
\setlength{\parindent}{0.7cm}% Espacio de la sangría
\setlength{\parskip}{14pt plus 1pt minus 1pt}% Separación entre párrafos

%\setlength{\parskip}{1ex plus 0.5ex minus 0.2ex}%

%==========================  Español venezolano =====================
%%Personalización de caption
\addto\captionsspanish{%
  \def\prefacename{Prefacio}%
  \def\refname{REFERENCIAS}%
  \def\abstractname{Resumen}%
  \def\bibname{REFERENCIAS}%{Bibliografía}%
  \def\chaptername{CAPÍTULO}%
  \def\appendixname{Apéndice}%{Anexo}
  \def\contentsname{ÍNDICE GENERAL}
  \def\listfigurename{LISTA DE FIGURAS}%Índice de Figuras\hspace*{10em}
  \def\listfigurenameTofC{LISTA DE FIGURAS}%Índice de Figuras
  \def\listtablename{LISTA DE TABLAS}%Índice de Tablas
  \def\indexname{Índice alfabético}%
  \def\figurename{Figura}%
  \def\tablename{Tabla}%
  \def\partname{Parte}%
  \def\enclname{Adjunto}%
  \def\ccname{Copia a}%
  \def\headtoname{A}%
  \def\pagename{Página}%
  \def\seename{véase}%
  \def\alsoname{véase también}%
  \def\proofname{Demostración}%
  \def\glossaryname{Glosario}
  }%



%==================================================================

%\setcounter{secnumdepth}{1}
%\setcounter{page}{4}
%\addtocounter{page}{4}%

\pagenumbering{roman}

\makeindex



%%%%%%%%%%%%%%%%%%%%%%%%%%%%%%%%%%%%%%%%%%%%%%%%%%%%%%%%%%%%%%%%%

\begin{document}
%\frontmatter
%===================================================================
%                            Primera Página
%================================== Portada =================================================
\renewcommand{\baselinestretch}{1.0}% Espaciado entre linea
\begin{titlepage}

\setlength{\unitlength}{1cm}%
\begin{picture}(5,5)(-5,0)
\put(-6,3){{
\begin{minipage}[h]{2cm}
%\includegraphics[width=2cm]{ucv.eps}
%\includegraphics[width=2cm]{newton.eps}
\end{minipage}}
}%
\put(-4,4){{
\begin{minipage}[h]{11cm}
\begin{center}
\begin{large}
\textbf{TRABAJO ESPECIAL DE GRADO}

%Facultad de Ingeniería

%Escuela de Ingeniería Eléctrica

\end{large}
\end{center}
\end{minipage}}
}%
\put(8,3){{
\begin{minipage}[h]{2cm}
%\includegraphics[width=2cm]{fi.eps}
%\includegraphics[width=2cm]{lagrange.eps}
\end{minipage}}
}%
\put(1,-12){{
\begin{minipage}[h]{8cm}
\begin{flushright}
\renewcommand{\baselinestretch}{1.0}% Espaciado entre linea
\begin{spacing}{1}
    Presentado ante la ilustre\\
Universidad Central de Venezuela\\
por el Br. Guillermo Raven\\
para optar al título de \\
Ingeniero Electricista.
\end{spacing}
\end{flushright}

\end{minipage}}
}%

\put(-1,-16){{
\begin{minipage}[h]{8cm}
Caracas, septiembre de 2022
\end{minipage}}
}%

\end{picture}
\begin{center}
\vspace{2.1cm}%
\begin{large}
\textbf{DESARROLLO DE UN PAQUETE EN PYTHON PARA EL POSICIONAMIENTO DE OBJETOS EN UNA ESCENA MEDIANTE VISIÓN ESTEREOSCÓPICA Y TÉCNICAS DE RECONOCIMIENTO BASADAS EN APRENDIZAJE AUTOMÁTICO
 }
\end{large}
\end{center}
\end{titlepage}

%%%%%%%%%%%%%%%%%%%%%%%%%%%%%%%%% Anteportada %%%%%%%%%%%%%%%%%%%%%%%%%%%%%%%%%%%%%%%%%
\newpage


\begin{titlepage}

\setlength{\unitlength}{1cm}%
\begin{picture}(5,5)(-5,0)
\put(-6,3){{
\begin{minipage}[h]{2cm}
%\includegraphics[width=2cm]{ucv.eps}
%\includegraphics[width=2cm]{newton.eps}
\end{minipage}}
}%
\put(-4,4){{
\begin{minipage}[h]{11cm}
\begin{center}
\begin{large}
\textbf{TRABAJO ESPECIAL DE GRADO}

%Facultad de Ingeniería

%Escuela de Ingeniería Eléctrica

\end{large}
\end{center}
\end{minipage}}
}%
\put(8,3){{
\begin{minipage}[h]{2cm}
%\includegraphics[width=2cm]{fi.eps}
%\includegraphics[width=2cm]{lagrange.eps}
\end{minipage}}
}%
\put(2,-12){{
\begin{minipage}[h]{8cm}
\begin{flushright}
\begin{spacing}{1}
    Presentado ante la ilustre\\
Universidad Central de Venezuela\\
por el Br. Guillermo Jose Raven Lusinche\\
para optar al título \\
de Ingeniero Electricista.
\end{spacing}
\end{flushright}

\end{minipage}}
}%

\put(-5.8,-8.5){{
\begin{minipage}[h]{11cm}
TUTORA: Profesora Tamara Pérez\\
\end{minipage}}
}%

\put(-1,-16){{
\begin{minipage}[h]{8cm}
Caracas, septiembre de 2022
\end{minipage}}
}%

\end{picture}
\begin{center}
\vspace{2.1cm}%
\begin{large}
\textbf{DESARROLLO DE UN PAQUETE EN PYTHON PARA EL POSICIONAMIENTO DE OBJETOS EN UNA ESCENA MEDIANTE VISIÓN ESTEREOSCÓPICA Y TÉCNICAS DE RECONOCIMIENTO BASADAS EN APRENDIZAJE AUTOMÁTICO
 }
\end{large}
\end{center}
\end{titlepage}
%===================================================================
% Una manera diferente, pero no permite muchas facilidades,
% de diseñar la primera página

%\title{\textbf{Título del Trabajo}}
%\author{Tu nombre}
%\date{\today}
%\maketitle

%======================= Constancia de Aprobación ===================
%\newpage
\begin{figure}
        \begin{center}
        %\centering
        %\includegraphics[height=23cm]{aprobacion.eps}

        \vspace{0.5mm}
        \label{Fig.aprobacion}
        \end{center}
        \end{figure}
\thispagestyle{empty}
%======================= Mención Honorífica =========================
\newpage
%\thispagestyle{empty}

\begin{figure}
        \begin{center}
        %\centering
        %\includegraphics[height=24cm]{mencion.eps}
        \vspace{0.5mm}
        \label{Fig.mencion}
        \end{center}
\end{figure}
\thispagestyle{empty}
%======================= Página de Dedicatoria ======================
\newpage%
\newenvironment{dedication}%
{\cleardoublepage \thispagestyle{empty} \vspace*{\stretch{1}}%
\begin{center} \em} {\end{center} \vspace*{\stretch{3}} }%
\begin{dedication}%
A mi madre Yuraima que con su amor, paciencia y esfuerzo me ha permitido alcanzar hoy un sueño mas y a mi hermana María que siempre me ha apoyado y creído en mi incluso cuando nadie mas creía en mi.
\end{dedication}%
\newpage
%==================================================================
\chapter*{RECONOCIMIENTOS Y AGRADECIMIENTOS}
%\markboth{Reconocimientos}{Reconocimientos}%
\addcontentsline{toc}{chapter}{RECONOCIMIENTOS Y AGRADECIMIENTOS}%
%\setlength{\parskip}{0.2cm}%
\hspace{0.5cm}En primer lugar quiero agradecer a mi madre Yuraima que fungió como motor de arranque para que culminara este trabajo, a la vez que siempre me apoyo todos esos años para que pudiese estudiar lo que me apasiona y por supuesto agradezco tenerla con vida a mi lado. A mi hermana Maria que siempre a creído en mí y me ha apoyado tanto en las buenas como en las malas y a mi familia que nos apoyó cuando nos enteramos de la enfermedad de mi madre.

\vspace{2mm}

\hspace{0.5cm}Una mención especial a mi tutora Tamara Pérez que a pesar de que pasara tanto tiempo para la culminación de este trabajo, me ayudo a permitirme el tiempo suficiente para terminarlo en reiteradas oportunidades, por este motivo le agradezco su empatia, sinceridad.

\vspace{2mm}

\hspace{0.5cm}Al profesor William La Cruz por ser el primero en mostrarme la realidad de lo que significaba estudiar esta profesión, el profesor Raúl Arreaza por enseñarme que en ocasiones es mejor decir ``No sé'' en lugar de mentir descaradamente para quedar bien, a Servandos que aunque ya no esté con nosotros siempre motivo a sus estudiantes a pensar en cada pequeño detalle de un circuito y a emprender nuevas ideas y al profesor Simón por demostrarme que la ingeniería en sí misma puede llegar a ser un arte.

\vspace{2mm}

\hspace{0.5cm}Agradezco a mis compañeros Enderson Omaña, Jesus Correa, Karla Arteaga y Alexis Fraudita por apoyarme y aconsejarme en todos estos años de carrera.%
%======================= Página de Resumen ==========================
\newpage
\renewcommand*{\abstract}{\begin{center}\end{center}}
%\begin{abstract}
\begin{spacing}{1}
\begin{center}%

\textbf{Guillermo Jose Raven Lusinche}

\begin{large}
\textbf{Desarrollo de un paquete en Python para el posicionamiento de objetos en una escena mediante visión estereoscópica y técnicas de reconocimiento basadas en aprendizaje automático}
\end{large}
\end{center}

%\noindent%
\justifying
\textbf{Tutora: Tamara Pérez. Tesis.
Caracas, Universidad Central de Venezuela. Facultad de Ingeniería.
Escuela de Ingeniería Eléctrica. Mención Electrónica, Computación y Control. Año 2022}

%\noindent
%\noindent \textbf{Resumen.-} Escribe acá tu resumen
Se creó el paquete de Python ``Py2vision'', el cual puede determinar las coordenadas homogéneas de un objeto detectado por la red neuronal YOLO V3. Además de dicha arquitectura de red, emplea el algoritmo semi-global block matching (SGBM) para determinar los mapas de disparidad y los refina con un filtro de mínimos cuadrados ponderados, para finalmente aplicar la segmentación binaria de Otsu dentro de cada ventana de detección y son estos píxeles segmentados, los que se utilizan para hallar la posición real del objeto en el espacio 3D. 
\vspace{0.2cm}

El paquete tiene la suficiente flexibilidad para que los módulos principales como lo son el estéreo y el de reconocimiento, sean independientes uno del otro. A su vez, cuenta con la documentación necesaria de cada función y un conjunto de tutoriales ejecutables en Google colab, que explican como utilizar los principales componentes, por este motivo es recomendable profundizar en el estudio de estas técnicas colaborando en el desarrollo de nuevas funcionalidades para dicho software, que utilicen diferentes estrategias estéreo y otros modelos de reconocimiento.

\vspace{0.2cm}

\textbf{Palabras Claves:} Python, paquete, mapa de disparidad, YOLO V3, coordenadas homogeneas, estereo, reconocimiento. \\[1ex]
\end{spacing}

%\underline{RESUMEN}
%
\thispagestyle{empty}%
%\input{resumen.tex}%
%\end{abstract}
\newpage
%====================== Páginas de Contenidos =====================
\addtocounter{page}{3}%
\setlength{\parskip}{3pt}% Separación entre párrafos
\tableofcontents%
\listoffigures%
\listoftables%

\newpage

%==================================================================
\chapter*{LISTA DE ACRÓNIMOS}%
%\markboth{Lista de Acrónimos}{Lista de Acrónimos}%
\addcontentsline{toc}{chapter}{LISTA DE ACRÓNIMOS}%
AA : Aprendizaje automático \\
RNA: Redes Neuronales Artificiales \\
VC: Visión por Computador\\
UCV: Universidad Central de Venezuela\\
ECM: Error Cuadrático Medio \\
CNN: Convolutional Neural Networks\\
ROS: Robot Operating System\\
SSD: Single Shot MultiBox Descriptor\\
ORB: Oriented FAST and Rotated BRIEF\\
DOPE: Deep Object Pose Estimation \\
3D: Tres dimensiones\\
2D: Dos dimensiones \\
FC: fully conected \\
%

%==================================================================
\justifying
\chapter*{INTRODUCCIÓN}\label{CAP:intro}
\setlength{\parskip}{14pt}% Separación entre párrafos
\addcontentsline{toc}{chapter}{INTRODUCCIÓN}%
%\markboth{Introducción}{Introducción}%

\pagenumbering{arabic}%
Se entiende por posicionamiento en el campo de visión por computador (VC) a todas aquellas técnicas que permiten extraer las coordenadas espaciales de un objeto a partir de su representación en 2 dimensiones. Entre las técnicas empleadas para el posicionamiento, aquellas que utilizan visión estéreo son ampliamente utilizadas en el campo de la robótica y los vehículos autónomos. Este hecho no es de extrañar, debido a que los inicios de esta tecnología datan de 1838 cuando Sir Charles Wheatstone publico un articulo en el que describía la visión estereoscópica \cite{Wheatstone1837}, en dicho articulo explica el como esta tecnología se basaba en la visión humana y en la separación de, aproximadamente, 65 mm que existe entre nuestros ojos. Estos reciben cada uno una imagen diferente que el cerebro une creando el efecto de tridimensionalidad. Por supuesto no tardo mucho hasta que decidieron aplicar esta teoría en el invento de moda de la época, las cámaras, y hoy en día el camino que se ha recorrido para permitir que un computador pueda interpretar coordenadas espaciales de una imagen nos ha permitido emplear las cámaras como sensores que interpretan el entorno que les rodea. 

En robótica se han desarrollado diversas metodologías de control basadas en sensores que dotan al robot de sentidos para percibir el mundo que le rodea. Algunas de las técnicas más usadas en el entorno industrial dependen de la odometría, que no es mas que el estudio de  la estimación de la posición de vehículos con ruedas durante la navegación. Para realizar esta estimación se usa información sobre la rotación de las ruedas mediante encoders con el fin de estimar cambios en la posición a lo largo del tiempo. Por otro lado, las técnicas de visión por computador en las décadas recientes, se han vuelto cada vez mas importantes debido al incremento en la capacidad de procesamiento y almacenamiento de los dispositivos electrónicos, además de la creciente necesidad realizar tareas más complejas con los autómatas, son flexibles a tal punto de que pueden ser implementadas con una o varias cámaras, las imágenes captadas son preprocesadas mediante filtros y luego con algoritmos, es posible determinar que objetos se encuentran en la imagen. Si se emplean metodologías basadas en visión estéreo, cámaras RGBD o láser, el robot gana la capacidad de localizarse en el entorno de trabajo o localizar objetos respecto al mismo.

En la actualidad las técnicas mas avanzadas de visión por computador (VC) utilizan las abstracciones conocidas como redes neuronales artificiales (RNA), las cuales suelen ser programadas en frameworks OpenSource basados en python, orientados al aprendizaje automático (AA). Algunos de los mas populares son Tensorflow o Keras por su compatibilidad con dispositivos embebidos  e incluso OpenCV para el preprocesamiento de las imágenes. Estas redes son capaces de aprender patrones en conjuntos de datos que se le suministra y dependiendo de como se le entrene pueden generalizar para datos que no se encuentren en el conjunto de datos de entrenamiento. En el campo de VC se traduce en detectar objetos para obtener la localización del mismo en una imagen o clasificar el tipo de objeto.

En la industria Venezolana no son muy comunes desarrollos en el campo de la localización de objetos en una escena, por lo que las técnicas de visión para el control no son muy comunes, de modo que las estrategias de control de robots suelen basarse en odometría o fusión de sensores, sumados a un conjunto de restricciones e instrucciones que controlen el movimiento de los autómatas. Por este motivo se propone desarrollar un paquete en Python que permita a programadores e ingenieros implementar sistemas de control cuya data sean imágenes provenientes de escenas estereoscópicas, los cuales podrán ser implementados en robots de uso domestico e industrial.%
%==================================================================

\chapter{MARCO REFERENCIAL}\label{CAP:marcoref}
En este capitulo se declarara el problema que fundamenta este trabajo, basado en el desarrollo de un paquete de python que permita obtener la localización de objetos en una escena con el fin de facilitar el desarrollo de sistemas de control basados en visión por computador. Además se expone la justificación, objetivos, alcance, factibilidad y antecedentes para este proyecto.

\section{Planteamiento del problema}
Las técnicas de control y automatización de procesos, aplicadas en el campo de la robótica, permiten realizar tareas sencillas en entornos estructurados donde se conoce el espacio de trabajo por completo y los objetos que interactúan en el mismo. No obstante, hoy en día los autómatas son capaces de realizar tareas complejas en espacios de trabajo dinámicos, es decir, entornos que varían en el tiempo, donde los eventos son de carácter estocástico. Sin embargo, para lograr dicha proeza es necesario expandir el conjunto de técnicas de control convencionales mediante la incorporación de nuevos algoritmos y estrategias.  

Para resolver problemas de robótica en entornos complejos, las mejores estrategias actuales son aquellas que dotan al robot de la capacidad de aprender y adaptarse a los cambios, estas técnicas de inteligencia artificial requieren de información del entorno la cual es extraída mediante algoritmos de VC que otorgan al autómata la habilidad de reconocer los objetos en su campo de visión y a partir de imágenes en 2D captar las posiciones relativas entre los objetos de interés y el robot. 

Existen diversas formas de conocer las posiciones de de objetos en escenas, una de ellas es la visión estéreo, aunque su implementación puede ser complicada si no se poseen los conocimientos apropiados, por este motivo se busca desarrollar un paquete que facilite la implementación de técnicas de posicionamiento de objetos en escenas mediante visión estéreo y reconocimiento basado en AA.
\section{Justificación}
El paquete para el posicionamiento basado en reconocimiento, facilitara la implementación de estrategias de control que empleen visión por computador a ingenieros y programadores. Y servirá para la comprensión de algunas de las técnicas de VC implementadas mediante tecnologías Open Source, como lo son Tensorflow, Keras y OpenCV.

Además de establecer algunos métodos para fortalecer los conocimientos en el área de control  que se imparten en la Escuela de Ingeniería Eléctrica, el estudio de tecnologías que se encuentran en auge usadas en el campo de inteligencia artificial aplicada a la robótica. 
\section{Objetivos}
\subsection{Objetivo general}
Desarrollar un paquete en Python para el posicionamiento de objetos en una escena empleando visión estereoscópica y técnicas de reconocimiento basadas en aprendizaje automático.
\subsection{Objetivo especifico}
\begin{itemize}
    \item Describir al menos dos técnicas de reconocimiento basadas en aprendizaje automático empleadas en el posicionamiento de objetos.
    \item Realizar el análisis de requerimientos necesarios del software a desarrollar.
    \item Definir la arquitectura del software.
    \item Definir los patrones de diseño aplicables.
    \item Seleccionar el modelo de aprendizaje automático para el reconocimiento de imágenes en vídeo.
    \item Implementar el sistema de reconocimiento con Tensorflow/Keras y OpenCV, en el lenguaje de programación Python.
    \item Implementar un algoritmo para el posicionamiento, basado en el reconocimiento de objetos en la escena.
    \item Diseñar un banco de pruebas para la validación del paquete.
    \item Elaborar un documento descriptivo del paquete con ejemplos de uso.
    \item Empaquetar el software para su distribución o instalación.
\end{itemize}
\section{Alcance y limitaciones}
El presente trabajo estará acotado en el desarrollo del paquete para el posicionamiento de objetos, en un entorno cerrado con la iluminación adecuada, para el correcto funcionamiento del modulo de reconocimiento. Se realizaran pruebas mediante cámaras de teléfono cuyos datos serán enviados a través de la aplicación gratuita de android IP webcam, para luego ser procesados en un computador o en un dispositivo embebido.
\section{Análisis de factibilidad}
Para la realización de este trabajo se cuenta con la documentación necesaria, además de casos de estudio donde lograron implementar sistemas similares en computadores y microcontroladores de gama baja y media. En caso de requerir la potencia de procesamiento que ofrece un servidor en la nube, se tiene un internet estable. El autor ya posee cierto nivel de conocimiento en las áreas de visión por computador y aprendizaje automático. Por otro lado se cuenta con el apoyo del ingeniero Carlos Gonzalez el cual se comprometió en prestar ayuda en el área de aprendizaje automático mediante el framework Tensorflow/Keras.

De ser necesaria alguna otra herramienta o equipo el autor cubrirá los gastos. Y la cantidad de tiempo que se propone se considera suficiente para la realización de este trabajo.
\section{Antecedentes}
En primer lugar se tiene el trabajo de fin de máster presentado en 2010 por
Martín Montalvo Martínez en la Universidad computense de Madrid, titulado "Técnicas de visión estereoscópica para determinar la estructura tridimensional de la escena"  \cite{MartinMM}.

En este trabajo se estudió la efectividad de diversos métodos de correspondencia estereoscópica, las técnicas utilizadas fueron comparadas mediante un estimador conocido como Error Cuadrático Medio (ECM), con este estimador se comparó el mapa de disparidad obtenido con cada uno de los métodos y el mapa de disparidad considerado como correcto "ground-truth". 
\\
El estudio se centro en la factibilidad para la implementación de sistemas estereoscópicos que han de operar en el exterior y bajo condiciones adversas, ya que las actividades de investigación planteadas por el grupo ISCAR en 2010 estaban orientadas en la navegación autónoma de vehículos. En estos vehículos el principal problema era el de la correspondencia estereoscópica, por este motivo el proyecto se oriento en la identificación de un algoritmo apropiado para el caso correspondiente.
\\
\\
La metodología empleada por el autor consistió en primer lugar, en realizar una revisión bibliográfica sobre los métodos descritos hasta la fecha, luego selecciono aquellos que encajaran con la problemática de estudio, para posteriormente implementarlos en el entorno de Matlab en su versión 2007b utilizando imágenes sintéticas y reales de las que poseía información sobre los mapas de disparidad. Luego evaluó su efectividad mediante el ECM y se obtuvo que de los métodos estudiados el que presenta mejores resultados es aquel basado en la segmentación y medida de similitud, además dicho método  se posiciona como el segundo más rápido al evaluar el promedio de tiempo. Sin embargo el problema que plantea radica en que los objetos existentes en la escena con una tonalidad uniforme y con distintos valores de disparidad son representados incorrectamente en el mapa final. Con este trabajo de maestría se logro comparar el comportamiento de varias metodologías que permiten obtener una representación tridimensional de una escena mediante visión estéreo.
\\
\\
En octubre Dembys, Gao, Shafiekhani, y Desouza (2019) presentaron un paper en una conferencia titulado "Detección de objetos y estimación de pose utilizando CNN (Convolutional Neural Networks) en hardware integrado para tecnología de asistencia" \cite{AssistiveTech}.
\\
\\
En este trabajo se desarrollo un algoritmo de visión estéreo el cual puede ser ejecutado en una placa Raspberry Pi 3 utilizando dos camaras RPi V2 como sensores. Este sistema se interconecta a un computador mediante el framework ROS (Robot Operating System) el cual posee una interfaz donde muestra el resultado final del sistema estereoscópico. El algoritmo utilizado esta compuesto por dos fases principales, la primera sería la fase de detección donde emplea una red neuronal convolucional (CNN) llamada MobileNet SSD (Single Shot MultiBox Descriptor) para reconocer y seguir los objetos de interés en una escena, mientras que en la fase de estimación de posición emplea correspondencias estéreo para reconstruir en 3D las coordenadas espaciales basadas en el algoritmo ORB (Oriented FAST and Rotated BRIEF) desarrollado por OpenCV. La motivación de esta investigación fue el desarrollo de equipamiento médico de bajo coste que pueda servir de apoyo para personas con discapacidades.
\\
\\
Los investigadores compararon sus resultados con la tecnología del estado del arte DOPE (Deep Object Pose Estimation) y llegaron a la conclusión de que es posible integrar redes ligeras para la detección de objetos en tiempo real y estimación de pose en tecnologías de asistencia a un bajo costo. Sin embargo, el algoritmo propuesto depende de la iluminación de la escena y la calidad de las características extraídas del objeto de destino, aunque los resultados demuestran que su enfoque también puede ser empleado en tareas de pick-and-place de brazos manipuladores.
%

%==================================================================
\chapter{MARCO TEÓRICO}\label{CAP:teor}
%\markboth{Tu Primer Capítulo}{Tu Primer Capítulo}%
En este capítulo se explicarán las etapas del proceso de visión estereoscópica, partiendo desde la adquisición de imágenes en una escena estéreo, la calibración de las cámaras que permite hallar los parámetros del sistema, la etapa de correspondencia de imágenes, algunos pasos adicionales como lo son la rectificación y la interpolación que mejoran el resultado final y concluyendo el proceso con la obtención de las coordenadas homogéneas de los píxeles. Posteriormente se definirán conceptos fundamentales de las redes neuronales, para luego precisar dos técnicas de reconocimiento que emplean aprendizaje automático. Por último se describirán conceptos fundamentales sobre los paquetes de \textit{Python}.
\section{Proceso de visión estereoscópica}
 En general la visión estéreo consiste en recuperar las características tridimensionales de una escena a partir de múltiples imágenes tomadas desde varios puntos de vista. De acuerdo con lo propuesto por Barnard y Fischler en 1982, las investigaciones sobre soluciones computacionales para la generalización del problema estéreo siguen un simple paradigma \cite{Barnard1982}, el cual involucra los siguientes pasos:
\subsection{Adquisición de imágenes}
Esta etapa engloba las condiciones externas del entorno, así como también el hardware seleccionado y como estos afectan al proceso estéreo \cite[p.~3]{Barnard1982}, por lo que a continuación se listan los factores más relevantes a tomar en cuenta:
\begin{itemize}
    \item La iluminación y el campo o dominio de aplicación, ya que; no es igual captar imágenes en entornos cerrados donde la iluminación puede ser constante, en comparación con entornos abiertos donde esta cambiará en el transcurso del día.
    \item Las condiciones climáticas pueden influir en la calidad de la imagen.
    \item La existencia de oclusión puede afectar al momento de interpretar las imágenes y posteriormente realizar satisfactoriamente el paso de correspondencia, ya que pueden no hallarse las coincidencias correspondientes entre píxeles.
    \item El posicionamiento relativo entre cámaras es un factor a tomar en cuenta, dado que este modifica los modelos utilizados en el sistema y su posterior procesamiento.
    \item El tiempo en que se toman las muestras de cada imagen, el cual dependerá de las especificaciones técnicas del hardware empleado, puesto que este posee limitaciones en cuanto a velocidad de transmisión y procesamiento.
    \item La resolución; que esta asociada al campo de aplicación.
    \item El campo de visión o \textit{field of view} (FOV), que no es más que el ángulo abarcable por el sensor de una cámara.
    \item El ruido (provocado por el sensor de la cámara) presente en las imágenes es reducido mediante un preprocesamiento para así mejorar el resultado final del sistema. 
\end{itemize}
\subsection{Modelado de la cámara (geometría del sistema)}
Es una representación de los atributos geométricos y físicos más importantes de las cámaras. Para poder representar los modelos de las cámaras se emplea el sistema de \textbf{Coordenadas Homogéneas} y la matriz homogénea, esta es una herramienta introducida por Forest en 1969 para resolver diferentes problemas de gráficos por computador a través de operaciones con matrices. Este tipo de transformaciones son empleadas para determinar en una sola matriz la posición y orientación de un objeto respecto a un sistema de referencia \cite{RSSFernando_homogeneusC}.

Cuando se emplean coordenadas homogéneas en imágenes, cada píxel tiene la siguiente representación:
\begin{equation}
(x, y) \Rightarrow
\begin{bmatrix}
x & y & 1
\end{bmatrix}^{T}
\end{equation}
Mientras que en el caso de una escena tridimensional la representación es:
\begin{equation}
(x, y, z) \Rightarrow
\begin{bmatrix}
x & y & z & 1
\end{bmatrix}^{T}
\end{equation}
Las traslaciones y rotaciones para las coordenadas homogéneas son operaciones lineales realizadas entre matrices, donde la nueva posición de un objeto, será el producto entre la posición previa y la matriz de transformación \cite{RSSFernando_homogeneusC}. La matriz de transformación homogénea definida por Forest es de dimensiones 4x4 y está compuesta a su vez por cuatro submatrices (Ecuación \eqref{homogeneusMatrix}).
\begin{equation}
    T = \begin{bmatrix}
        \begin{array}{c|c}
                rotaci\acute{o}n & traslaci\acute{o}n\\
                \hline
                perspectiva & escalado
        \end{array}
        \end{bmatrix}
        =
        \begin{bmatrix}
        \begin{array}{c|c}
                3 x 3 & 3 x 1\\
                \hline
                1 x 3 & 1 x 1
        \end{array}
        \end{bmatrix}
\label{homogeneusMatrix}
\end{equation}
No obstante la matriz de transformación en el plano imagen es de dimensiones 3 x 3 y tiene la siguiente representación:
\begin{equation}
    T = \begin{bmatrix}
        \begin{array}{c|c}
                rotaci\acute{o}n & traslaci\acute{o}n\\
                \hline
                perspectiva & escalado
        \end{array}
        \end{bmatrix}
        =
        \begin{bmatrix}
        \begin{array}{c|c}
                2 x 2 & 2 x 1\\
                \hline
                1 x 2 & 1 x 1
        \end{array}
        \end{bmatrix}
\end{equation}
Cuando se emplean coordenadas homogéneas no solo es más eficaz a nivel computacional, ya que permite efectuar cambios en la posición y orientación mediante productos matriciales, incluso se pueden convertir en coordenadas convencionales mediante una simple operación de división tal que:
\begin{align}
\begin{bmatrix}
x & y & w
\end{bmatrix}^{T} \Rightarrow (x/w, y/w)\label{homogeneus2d}\\
\begin{bmatrix}
x & y & z & w
\end{bmatrix}^{T} \Rightarrow (x/w, y/w, z/w)\label{homogeneus3d}
\end{align}
La ecuación \eqref{homogeneus2d} se utiliza para convertir un píxel de una imagen 2D que se encuentra en coordenadas homogéneas a cartesianas, mientras que la ecuación \eqref{homogeneus3d} se utiliza en el caso de tres dimensiones.
\\
Para poder estudiar un modelo de múltiples cámaras, es menester comprender como se aplican las coordenadas homogéneas en el caso de una sola cámara. Por este motivo a continuación se presentará el \textbf{Modelo de Pinhole}, el cual consiste en que cada punto de un objeto situado en el entorno de trabajo (espacio tridimensional) se proyecta en un punto de un plano denominado plano imagen ó plano de proyección. En la Figura \ref{pinholeModel} \textbf{PP} es el plano de proyección y \textbf{COP} se refiere al centro de proyección o centro óptico de la cámara. El punto en la escena 3D se denota como $p^{M}$, dicho punto se encuentra en las coordenadas del mundo $p^{M}$ $(x, y, z)$ y su proyección en el plano de imagen se denota como $p^{I}$ $(x', y')$ y corresponde con la intersección entre la línea que une $p^{M}$ y \textbf{COP} con el plano de imagen, además a la distancia entre el plano de proyección y \textbf{COP} se le conoce como distancia focal. 
\begin{figure}[H]
    \centering
    \includegraphics[scale=0.5]{Recursos/pinholeModel.jpg}
    \caption[Modelo de cámara Pinhole.]{Modelo de cámara Pinhole. {\footnotesize Fuente: El Autor}}
    \label{pinholeModel}
\end{figure}
La razón del porqué en el modelo presentado en la Figura \ref{pinholeModel} el plano de proyección se encuentra por delante del lente (a pesar de que en la realidad la luz ingresa a través del punto focal y luego impacta con el plano de imagen) es debido a que matemáticamente es más conveniente, porque de esta forma las imágenes no son invertidas.

Utilizando el teorema de Tales es posible determinar las proyecciones de los puntos en el plano de forma tal que se cumple la siguiente ecuación:
\begin{equation}
    (X, Y, Z) \longrightarrow (-d\frac{X}{Z}, -d\frac{Y}{Z}, -d)  \label{convert3Dto2Dpinhole}
\end{equation}
La operación realizada por la ecuación \eqref{convert3Dto2Dpinhole}, permite proyectar los puntos del espacio 3D en el plano de imagen, dicha operación puede ser realizada mediante coordenadas homogéneas de la siguiente forma:
\begin{align}
            \begin{bmatrix}
            1 & 0 & 0 & 0\\
            0 & 1 & 0 & 0\\
            0 & 0 & 1/f & 0
            \end{bmatrix}
            \begin{bmatrix}
            x\\
            y\\
            z\\
            1
            \end{bmatrix}
            =
            \begin{bmatrix}
            x\\
            y\\
            z/f\\
            \end{bmatrix} \Rightarrow \left(-f\frac{X}{Z}, -f\frac{Y}{Z}\right)\label{convert3Dto2Dperspective}
\end{align}
A la forma en la que se proyecta en la ecuación \eqref{convert3Dto2Dperspective} se le conoce como proyección de perspectiva \cite{Szeliski2022} y esta llega al mismo resultado que el de la ecuación \eqref{convert3Dto2Dpinhole}, ya que $f$ y $d$ son la distancia focal. Aunque la transformación presente en la ecuación \eqref{convert3Dto2Dperspective} es válida, el enfoque más común para determinar el modelo de una cámara utiliza al menos 11 parámetros, a continuación se detallara la metodología para hallar dichos parámetros:
\subsubsection{Calibración de la cámara}\label{calibration_section}
Para modelar una cámara y así localizar un objeto del mundo real (espacio tridimensional) a partir de imágenes es necesario recuperar información que se pierde al pasar del espacio 3D al plano de imagen. La información perdida corresponde con los parámetros que intervienen en el proceso de formación de imágenes, los cuales se definen como:
\begin{itemize}
\item \textbf{Parámetros intrínsecos:} son los que describen la geometría y óptica del sensor. Engloban el proceso desde que un rayo alcanza la lente del objetivo hasta que impresiona un elemento sensible  \cite{RSSFernando_homogeneusC}.
\item \textbf{Parámetros extrínsecos:} son aquellos que definen la orientación y posición de la cámara respecto a un sistema de coordenadas conocido al que se llama sistema del mundo. Se representa con una matriz 3 x 4, donde las primeras 3 columnas corresponden con la submatriz de rotación y la última es la submatriz de translación. Posee 3 grados de libertad para la orientación y 3 que definen el desplazamiento  \cite{RSSFernando_homogeneusC}.
\end{itemize} 
Los parámetros intrínsecos y extrínsecos de una cámara, permiten convertir un punto que se encuentre en el sistema de referencia del mundo $p^{M}$ al plano de imagen $p^{I}$ y viceversa. A continuación se presenta la forma en la que se realiza dicha transformación:
\begin{align}
    p^{I} = M p^{w} \\
    p^{I} = M \begin{bmatrix}
            x & y & z & 1
            \end{bmatrix}^{T} \\
    p^{I} = K \begin{bmatrix}R & T\end{bmatrix} \begin{bmatrix}
            x & y & z & 1
            \end{bmatrix}^{T}  \label{point_projection}
\end{align}
En la ecuación \eqref{point_projection} la matriz \textbf{K} de dimensiones 3x3 representa los parámetros intrínsecos, por lo que la matriz que involucra la rotación y traslación del sistema corresponde con los parámetros extrínsecos \cite[p~56]{Szeliski2022}. Existen diversos métodos de calibración que permiten hallar algunos o todos los parámetros necesarios para realizar la transformación de coordenadas 3D a 2D, pero de todas las estrategias, la más empleada, es la propuesta por Zhengyou Zhang \cite{Zhang2000}, la cual sigue los siguientes pasos para obtener ambas matrices:
\begin{enumerate}
\item Se imprime un patrón (es común que el patrón sea en forma de tablero de ajedrez) y se adhiere sobre una superficie plana.
\item Se capturan $n$ imágenes del plano de ajedrez, donde $n$ suele variar entre 15 y 20.
\item Se detectan las esquinas internas en cada imagen de cada tablero. En el caso de ser un patrón circular se detectaran los centros de cada punto de calibración.
\item Se estiman los parámetros intrínsecos y extrínsecos usando la solución cerrada propuesta por Zhengyou Zhang \cite{Zhang2000}
\item Se refina el cálculo de los parámetros incluyendo la distorsión generada por la lente de la cámara. Es importante recalcar, que este es un proceso iterativo, emplea el criterio de máxima verosimilitud para así alcanzar el valor más óptimo, por lo que puede darse el caso de que dependiendo del número de imágenes o las posiciones en las que fue capturado el tablero, la solución no converja a un valor que minimice el error.
\end{enumerate}
\subsubsection{Modelo de dos cámaras}
El modelo de dos cámaras o \textbf{Modelo Estéreo}, es una extrapolación del \textbf{Modelo de Pinhole}. En general se pueden tener 3 casos en función de cómo se encuentren orientados los ejes de ambas cámaras: disposición convergente, disposición alineada y la disposición divergente, la tercera no permite llevar a cabo un análisis estéreo \cite{RSSFernando_homogeneusC}. En la figura \ref{epipolar_geometry}, se puede observar el modelo geométrico de dos cámaras cuando sus ejes ópticos son convergentes, a continuación se listan cada uno de los elementos relevantes:
\begin{itemize}
\item \textbf{Linea base:} es la distancia que separa los dos centros ópticos de ambas cámaras. En la Figura \ref{epipolar_geometry} se asocia a la línea naranja y se identifica con la letra $B$.
\item \textbf{distancia focal:} la distancia entre el plano de imagen y los centros ópticos. Se denota $f$.
\item\textbf{El punto P:} corresponde con la distancia \textbf{Z} en las coordenadas de los sistemas de cámaras, siempre es medida hasta el centro de proyección ($COP$ ó $O$) de cada cámara. 
\item \textbf{Plano epipolar de un punto P en el espacio:} en la Figura \ref{epipolar_geometry} es el plano conformado por los puntos $P$, $O_{1}$ y $O_{2}$.
\item\textbf{Línea epipolar:} hay una por cada punto ($p$ y $p'$), en el caso de la Figura \ref{epipolar_geometry}, corresponde con la intersección del plano epipolar y cada plano de imagen, a su vez, esta contiene la proyección del punto y el epipolo.
\item\textbf{Epipolo:} Hay uno por cada cámara ($e$ y $e'$). Es la proyección sobre una cámara del centro óptico de la otra cámara. En el epipolo confluyen todas las líneas epipolares. En el caso de una disposición alineada no se representan los epipolos, ya que estos se encuentran en el infinito.
\end{itemize}
\begin{figure}[H]
    \centering
    \includegraphics[scale=0.3]{Recursos/epipolar_geometry.png}
    \caption[Disposición convergente de la geometría epipolar.]{Disposición convergente de la geometría epipolar. {\footnotesize Fuente: https://es.acervolima.com/python-opencv-geometria-epipolar/ \cite{acervolima}}}
    \label{epipolar_geometry}
\end{figure}
El potencial de la \textbf{Geometría Epipolar} está en que para todo píxel $p$ solo puede corresponder una proyección $p'$, la cual siempre estará situada a lo largo de una línea epipolar, definida por el epipolo $e'$ y la proyección $p'$ \cite[p~754]{Szeliski2022}. Esta restricción permite reducir considerablemente el espacio de búsqueda de proyecciones correspondientes en la otra imagen a una sola dimensión. Para poder aprovechar las ventajas de la geometría epipolar a nivel computacional, se utiliza el álgebra de la siguiente forma: asumiendo que un punto \textbf{P} en el sistema del mundo se proyecta en ambos planos de imagen como $p$ y $p'$ la representación de una proyección respecto a la otra es tal que
\begin{equation}
    p' = Rp + T
\end{equation} 
donde $R$ es la matriz de rotación que indica los ángulos de rotación de una cámara respecto a otra y $T$ es la matriz de traslación entre ambas ó la línea base del sistema. Si se aplica el producto cruz de $T$ en ambos lados de la expresión se tiene que:
\begin{equation}
    T \times p' = T \times Rp + T \times T
\end{equation}
De la expresión previa $T \times p'$ es perpendicular al plano epipolar y $T \times T$ es 0, de modo que:
\begin{align}
    T \times p' = T \times Rp
\end{align}
Aplicando el producto escalar de $p'$
\begin{align}
    p'\cdot(T \times p') = p' \cdot(T \times Rp)\\
    0 = p' \cdot(T \times Rp)
\end{align}
Y substituyendo $E$ = $T \times R$, se tiene que:
\begin{equation}
    p'^{T} \cdot(Ep) = 0 \label{esential_eq}
\end{equation}
En la expresión \eqref{esential_eq} $E$ es conocida como la matriz esencial \cite[p~257]{Hartley2004} y esta depende únicamente de la disposición del sistema de adquisición de datos, dicha ecuación indica que si se tiene un punto en una imagen de un conjunto de cámaras calibradas, existirá una línea en la otra imagen sobre la cual reposara la proyección de dicho punto. En una disposición de cámaras alineadas la única diferencia en la matriz esencial es que la matriz $R$ es igual a la identidad, por lo que el producto vectorial $T \times R$ se simplifica.
\\
\\
Por medio de la matriz esencial es posible reducir el problema de correspondencia en el modelo estéreo; sin embargo, esta metodología presenta el inconveniente de que ambas cámaras sean calibradas, por lo tanto para conocer la geometría epipolar sin calibrar el sistema, se selecciona un conjunto de puntos y sus correspondencias en cada cámara, cuya representación en el plano de imagen coincide con la expresión \eqref{point_projection} y a partir de dicha expresión se tiene que un punto en el plano de imagen respecto al marco referencial de la cámara está dado por:
\begin{equation}
    p^{I} = Kp^{c}
\end{equation}
La matriz \textbf{K} es invertible por lo que es posible expresar el caso contrario de la siguiente forma
\begin{equation}
    p^{c} = K^{-1}p^{I} \label{point_in_camera_frame}
\end{equation}
La expresión \eqref{point_in_camera_frame} aplicada a ambas cámaras sería:
\begin{align}
    p_{c,l} = K^{-1}_{l}p^{I}_{l}\\
    p_{c,r} = K^{-1}_{r}p^{I}_{r}
\end{align}
Y para poder relacionar las dos ecuaciones previas, se hace uso de la expresión \eqref{esential_eq} que relaciona ambas proyecciones de un punto, de modo que:
\begin{align}
    p_{c,r}^{T} E p_{c,l} = 0\\
    (K^{-1}_{r}p^{I}_{r})^{T}E(K^{-1}_{l}p^{I}_{l}) = 0\\
    (p^{I}_{r})^{T}((K^{-1}_{r})^{T}EK^{-1}_{l})p^{I}_{l} = 0\\
    F = (K^{-1}_{r})^{T}EK^{-1}_{l} \label{fundamental_matrix} \\
    (p^{I}_{r})^{T}Fp^{I}_{l} = 0 \label{fundamental_eq}
\end{align}
En la expresión \eqref{fundamental_eq} $F$ corresponde con la matriz fundamental la cual se define en la ecuación \eqref{fundamental_matrix}, esta depende únicamente de la matriz esencial y los parámetros internos de ambas cámaras, además posee las siguientes propiedades:
\begin{itemize}
    \item Si $p^{T}Fp'$ = 0, se tiene que la expresión $l$ = $Fp'$ es la línea epipolar en la imagen que contiene a $p$, la cual esta asociada a la proyección $p'$. Similar al caso previo $l'$ = $F^{T}p$ es la línea epipolar en la imagen que contiene a $p'$, la cual se asocia a la proyección $p$.
    \item Los epipolos en ambos planos de imagen pueden ser hallados resolviendo las expresiones $Fp'$ = 0 y $F^{T}p$ = 0.
    \item La matriz fundamental $F$ posee dimensiones 3 x 3 y es singular.
\end{itemize}
En síntesis la matriz fundamental \cite[p~241]{Hartley2004} relaciona las coordenadas de los píxeles en dos vistas o planos de imagen y es más general que la matriz esencial debido a que remueve la necesidad de requerir una calibración previa, ya que es posible estimar la matriz fundamental, a partir de las correspondencias de las coordenadas de los píxeles y así reconstruir la geometría epipolar sin conocer los parámetros intrínsecos y extrínsecos.
\subsubsection{Rectificación del modelo} \label{rectification_section}
De acuerdo con Richard I. Hartley el método de rectificación de imágenes, es un proceso que remuestrea un par de imágenes estéreo tomadas desde diferentes puntos de vista para así producir un par de proyecciones epipolares coincidentes. Estas son
proyecciones en las que las líneas epipolares corren paralelas al eje x, y en consecuencia las disparidades estarán únicamente en la dirección de dicho eje. Su método se basa en el análisis de la matriz fundamental de Longuet-Higgins la cual describe la geometría epipolar. Esta aproximación utiliza métodos basados en proyección geométrica para así determinar un par de transformaciones en 2D que permitan convertir un sistema de disposición convergente en uno de disposición alineada \cite{Hartley1999}.
\subsection{Correspondencia de las imágenes}
La finalidad de la correspondencia, es calcular el grado de disparidad existente entre las localizaciones de las proyecciones en cada imagen, para así convertir dicha disparidad en la profundidad y recuperar la información perdida en el proceso de captación de imágenes, la disparidad puede ser representada de la siguiente forma:
\begin{equation}
d = x_{l} - x_{r} = \frac{f\cdot B}{z_{p}}
\end{equation}
Existen dos formas de correspondencia la dispersa y la densa. La primera esta basada en extraer un conjunto de puntos característicos (puntos de interés) en ambas imágenes, los cuales son tomados mediante detectores de bordes o detectores de esquinas, luego se buscan las ubicaciones correspondientes de dichos puntos o áreas en la otra imagen, estas técnicas surgieron debido al limitado recurso computacional y al deseo de reducir las respuestas producidas por algoritmos estéreo. A pesar de que se siguen utilizando algoritmos de correspondencia dispersa, la mayoría están basados en correspondencia densa, la cual calcula la disparidad en toda la escena en lugar de una sección de la misma, sus aplicaciones pueden ser el renderizado o modelado 3D \cite{Szeliski2022}. 
\\
\\
De acuerdo con lo propuesto por Scharstein y Szeliski \cite{Scharstein2002}, el esquema de clasificación y taxonomía para los algoritmos de correspondencia densa, consiste en un conjunto de ''bloques de construcción'' algorítmicos, los cuales pueden estar conformados por un subconjunto de los siguientes cuatro pasos:
\begin{enumerate}
    \item \textbf{Cálculo de costos de correspondencia:} Es una medida de similitud que compara los valores (intensidades) de los píxeles en orden de determinar que tan probable es que correspondan. La mayoría de las técnicas de correspondencia basadas en píxeles incluyen las funciones de coste presentadas en la tabla \ref{matchingCostTypes}, sin embargo se han utilizado funciones de coste como la correlación cruzada normalizada, histogramas e incluso gradientes, la elección de la función de coste, depende del entorno de aplicación y los recursos computacionales \cite[p~3]{Scharstein2002}.
    \begin{table}[H]
    \centering
    \renewcommand{\arraystretch}{2}
    \caption{Funciones de costos de correspondencia comunes}
    \label{matchingCostTypes}
    \begin{tabular}{|c|c|}
    \hline
    Nombre de la métrica & Función respecto a (x, y) \\
    \hline
    Suma de diferencias al cuadrado (SSD) & $\sum_{m=0}^{M-1}sum_{n=0}^{N-1}(I_{1}(x+m, y+n) - I_{2}(m, n))^{2}$    \\
    \hline
    Suma de diferencias absoluto (SAD)    & $\sum_{m=0}^{M-1}sum_{n=0}^{N-1}\lvert I_{1}(x+m, y+n) - I_{2}(m, n)\rvert$ \\
    \hline
    \end{tabular}
    \end{table}
    En la tabla \ref{matchingCostTypes} las funciones de coste aplicadas sobre las imágenes $I_{1}$ y $I_{2}$ tienen como objetivo hallar en ambas imágenes las ventanas que obtengan el menor resultado, ya que estas serán las que mayor similitud tendrán.
    \item \textbf{Agregación de costos (soporte):} Es un paso muy ligado al paso previo, puesto que engloba la sumatoria o el promedio de las funciones de coste mediante una ventana o función de soporte \cite[p~4]{Scharstein2002}. 
    \item \textbf{Cálculo de disparidad y optimización:} Por lo general como se mencionó en el paso 1, se seleccionan los píxeles para calcular la disparidad, asociando aquellos que menor coste obtengan, aunque este hecho puede ser cierto para los métodos locales los cuales suelen enfocar sus esfuerzos en el paso 1 y 2, no obstante cuando se usan métodos de correspondencia globales la mayoría del trabajo es realizado en este paso, debido a que estos son formulados en un marco de referencia enfocado en la minimización de energía, cuyo objetivo es hallar una función que minimice la energía global \cite[p~5]{Scharstein2002}.
    \item \textbf{Refinamiento de disparidad:} En aplicaciones de navegación robotizada o seguimiento de personas puede ser un paso opcional, sin embargo; cuando se quiere modelar mediante imágenes es un paso esencial el cual tiene diversos enfoques, desde la interpolación para reducir las disparidades desconocidas que se encuentren en las vecindades de disparidades conocidas, cálculos de sub-píxeles que pueden ser determinados incluso con el método iterativo del descenso al gradiente el cual ajustara los costos de correspondencia a valores de disparidad discretos \cite[p~6]{Scharstein2002}.
\end{enumerate}
\subsection{Determinación de la distancia (profundidad)}
Una vez que se ha hecho corresponder los elementos que aparecen en la imagen izquierda con los elementos en la imagen derecha, la determinación de la profundidad es un proceso relativamente sencillo, reduciéndose a una simple triangulación. Sin embargo, en algunas ocasiones cuando se intenta hallar la distancia a la que se encuentra una característica, se presentan algunas dificultades debidas a una falta de precisión o una escasa fiabilidad cuando se intentó encontrar la correspondencia.
\\
La triangulación dependerá de la geometría epipolar del modelo, aunque el cálculo puede ser reducido, si se emplea un modelo de ejes alineados o se rectifica el sistema.  En las Figuras \ref{estereoSystemParallel} y \ref{estereoSystemParallel3D}, se representa un sistema de ejes alineados, el cual tiene la ventaja de que para calcular las coordenadas de las proyecciones del punto P sobre cada una de las cámaras, basta con utilizar el teorema de triángulos similares, de tal forma que su posición en el espacio estará dada por: 
\begin{align}
x_{p} = \frac{x_{l}\cdot B}{x_{l} - x_{r}}\\
y_{p} = \frac{y_{l}\cdot B}{x_{l} - x_{r}}\\
z_{p} = \frac{f\cdot B}{x_{l} - x_{r}}
\end{align}
La diferencia $x_{l} - x_{r}$ es conocida como disparidad y es inversamente proporcional a la profundidad o distancia.
\begin{figure}[H]
     \centering
     \begin{subfigure}[b]{0.4\textwidth}
        \centering
        \includegraphics[scale=0.4]{Recursos/stereoGeometry.jpg}
        \caption[Representación del modelo de ejes alineados en el plano.]{Representación del modelo de ejes alineados en el plano. {\footnotesize Fuente: El Autor}}
        \label{estereoSystemParallel}
     \end{subfigure}
    \hspace{1em}
     \begin{subfigure}[b]{0.4\textwidth}
         \centering
        \includegraphics[scale=0.5]{Recursos/stereoGeometry3D.jpg}
        \caption[Representación del modelo de ejes alineados en el espacio.]{Representación del modelo de ejes alineados en el espacio. {\footnotesize Fuente: \textit{Computational Stereo} \cite{Barnard1982}}}
        \label{estereoSystemParallel3D}
     \end{subfigure}
     \hfill
\caption{Modelo estéreo de ejes alineados}
\label{stereoMODEL}
\end{figure}
\section{Fundamentos del aprendizaje profundo o deep learning}
El aprendizaje profundo, es una rama del aprendizaje automático ó \textit{machine learning}, que se caracteriza por poseer un conjunto de datos de entradas, ejemplos de una salida esperada por el sistema y una forma de medir cuando un algoritmo está realizando adecuadamente su trabajo, este último elemento es necesario, ya que permite determinar la distancia entre la salida actual del algoritmo y la salida esperada, la medición de la salida es usada como una señal de realimentación para así ajustar la forma en que el algoritmo funciona. A este ajuste se le conoce como ``aprendizaje''. Cabe destacar que cuando a un modelo de aprendizaje automático se le suministran las salidas esperadas pasa a ser parte de una sub-rama del aprendizaje profundo conocido como aprendizaje supervisado, pero no es una característica obligatoria del aprendizaje profundo suministrar la salida esperada, a esta otra sub-rama se le conoce como aprendizaje no supervisado.
\\
La idea base de este sub-campo del \textit{machine learning}, nace del interés de replicar en sistemas computacionales el comportamiento de las neuronas presentes en los seres vivos, las cuales son responsables del aprendizaje y sus inicios datan de 1958 cuando Rosenblatt desarrollo \textbf{El Perceptron} (Ver Figura \ref{perceptron}).
\begin{figure}[H]
    \centering
    \includegraphics[scale=0.35]{Recursos/perceptron.png}
    \caption[Comparación de una neurona biológica con una neurona artificial.]{Comparación de una neurona biológica con una neurona artificial. {\footnotesize Fuente: Introducción a las redes neuronales artificiales \cite{SierraRamosJM}}}
    \label{perceptron}
\end{figure}
En comparación con otras técnicas de aprendizaje automático, que normalmente se basan en varias
etapas de preprocesamiento para extraer características en las que se pueden construir clasificadores, el enfoque de aprendizaje profundo
generalmente se entrena de un extremo a otro, yendo directamente desde píxeles sin procesar hasta los resultados finales deseados \cite{Szeliski2022}.
\subsection{Elementos de las redes neuronales}
\subsubsection{Pesos y capas}
Una red neuronal profunda extiende la idea del perceptron (Ver Figura \ref{perceptron}) que utiliza una única neurona, a un grafo compuesto por miles de neuronas interconectadas \cite[p~270]{Szeliski2022}. En la Figura \ref{NeuralNetworkArq} cada circulo corresponde con una neurona cuya salida estará dada por la siguiente expresión:
\begin{equation}
    s_{i} = w_{i}^{T} x_{i} + b_{i} \label{ponderedSum}
\end{equation}
el resultado de las sumas ponderadas (ecuación \eqref{ponderedSum}) de cada neurona pasa por una función de activación no linear que redistribuirá los valores en un rango acotado:
\begin{equation}
    y_{i} = h(s_{i})
\end{equation}
En la ecuación \eqref{ponderedSum}, los valores de $x_{i}$ son las entradas de la i-esima neurona, $w_{i}$ y $b_{i}$ son parámetros que se van ajustando en el proceso de aprendizaje y corresponden con los pesos y el bias respectivamente. 
\begin{figure}[H]
    \centering
    \includegraphics{Recursos/NeuralNetwork.jpg}
    \caption[Arquitectura de una red neuronal multicapas.]{Arquitectura de una red neuronal multicapas. {\footnotesize Fuente: Notas sobre pronóstico del Flujo de Tráfico en la Ciudad de Madrid \cite{mañas}}}
    \label{NeuralNetworkArq}
\end{figure}
A las capas de la Figura \ref{NeuralNetworkArq} se les conoce como \textit{fully conected} (FC), debido a que todas las entradas de una capa están conectadas a todas sus salidas. Dado que las redes neuronales suelen estar organizadas en capas consecutivas, es posible agrupar cada neurona dentro de una capa en un vector, cuya combinación lineal se escribe como:
\begin{equation}
    s_{l} = w_{l}^{T} x_{l}
\end{equation}
donde $x_{l}$ son las entradas a la capa $l$, $W_{l}$ es una matriz de peso y $s_{l}$ es la suma ponderada, a la que se aplica una función de activación:
\begin{equation}
    x_{l+1} = y_{l} = h(s_{l})
\end{equation}
Una red que consiste únicamente en capas FC, se le conoce como perceptron multicapa. 
\subsubsection{Funciones de activación}
Las funciones de activación se usan para propagar la salida de los nodos de una capa hacia la siguiente capa. Además este tipo de funciones permiten incorporar al modelo no linealidades, que permitirán que este se ajuste a cualquier distribución de datos. Algunas de las funciones de activación más comunes son:
\begin{itemize}
    \item \textbf{función logística o sigmoidea:} Este tipo de funciones permiten mitigar el efecto de \textit{outliers} (valores atípicos que son numéricamente distantes del resto de los datos) en el entrenamiento del modelo. La imagen de este tipo de funciones suele estar contenida en los intervalos [0, 1] \cite{sharma_2021}. Por lo que valores muy extremos siempre estarán cerca de los límites del intervalo de esa imagen. Su ecuación está dada por:
    \begin{equation}
        h(x) = \frac{1}{1 + e^{-x}}
    \end{equation}
    \item \textbf{Tangente hiperbólica:} Es una función no lineal cuyo rango normalizado se encuentra entre [-1, 1] y su ventaja es que puede manejar más fácilmente los números negativos. Su ecuación está dada por:
    \begin{equation}
        h(x) = \frac{e^{x} - e^{-x}}{e^{x} + e^{-x}}
    \end{equation}
    \item \textbf{ReLu o rampa:} Es una de las funciones más utilizadas en redes convolucionales y en redes profundas debido a que con ella se logran mejores resultados que con la sigmoidea y la hiperbólica, ya que en el procesamiento de imágenes los valores negativos no son importantes y por lo tanto se establecen en 0. Pero los valores positivos después de la convolución deben pasar a la siguiente capa. En cambio si se emplean la hiperbólica o la sigmoidea, la información se pierde ya que ambas funciones modificarán las entradas a un rango muy cerrado. Su ecuación esta dada por:
    \begin{equation}
        h(x) = max(0,x)
    \end{equation}
\end{itemize}
\subsubsection{Funciones de error}
Como se ha podido observar el proceso de aprendizaje de una red neuronal es de carácter iterativo, ya que, lo que se busca es ajustar un modelo a un conjunto de datos. Las funciones de error permiten medir la distancia a la que se encuentra un modelo del objetivo, de esta forma es posible tomar medidas adecuadas en la siguiente iteración, para así reducir el error hasta alcanzar valores mínimos \cite[p~280]{Szeliski2022}. Dependiendo de la tarea a realizar se utilizan diferentes funciones de error. 
\\
Por ejemplo, en el caso de redes que utilicen como datos de entrada mapas de profundidad (mapas de disparidad) ó imágenes sin ruido, se suele realizar una regresión matemática que ajuste al modelo, por lo que es común utilizar la norma L2 como función de error
\begin{equation}
    E(w) = \sum_{n} E_{n}(w) = - \sum_{n} ||y_{n} - t_{n}||^{2}
\end{equation}
donde $y_{n}$ es la salida de la red para la muestra n y $t_{n}$ es el valor objetivo. Sin embargo, si son pocos los \textit{outliers} en los datos de entrenamiento, o si los errores graves no son tan dañinos como para merecer una penalización cuadrática, se pueden utilizar normas más robustas como la L1
\begin{equation}
    E(w) = \sum_{n} E_{n}(w) = - \sum_{n} ||y_{n} - t_{n}||
\end{equation}
Cabe resaltar que no son las únicas funciones de error utilizadas en el campo de visión estéreo, pues existen variaciones, como en el trabajo realizado por Sizhang Dai y Weibing Huang en marzo del 2020 \cite{dai2020atvsnet} donde usan el error absoluto medio.
\subsubsection{Técnicas para la optimización de la red}
Cuando se entrena una red se emplean diversas técnicas que mejoran los resultados de la misma, o permiten que las redes neuronales no sufran de \textit{overfitting} (sobre ajuste) para así poder generalizar mejor. A continuación se detallaran algunas de las técnicas más empleadas así como lo es el \textit{dropout} y la normalización del lote.
\begin{itemize}
    \item \textbf{Regularización:} es una técnica que surge cuando se optimizan los pesos dentro de una red neuronal, los pesos se vuelven más pequeños, a este fenómeno se le conoce como decaimiento de los pesos y es un problema debido a que al pasar por la función de activación un valor de suma ponderada muy cercano a 0, es mucho más difícil para el algoritmo del descenso al gradiente desplazar el valor de ese peso a su óptimo. En la práctica, valores muy pequeños se traducen a coeficientes con muchos decimales y esto es sinónimo de \textit{overfitting}, para solucionar esto se aplica regularización a la función de error, lo que penaliza en mayor medida a los pesos con largos coeficientes \cite[p~275]{Szeliski2022}. A continuación se presentan las regularizaciones L1 y L2 respectivamente.
    \begin{align}
          E(w)_{L1} = - \sum_{n} ||y_{n} - t_{n}|| + \lambda \sum_{i} |w_{i}|\\
          E(w)_{L2} = - \sum_{n} ||y_{n} - t_{n}|| + \lambda \sum_{i} w_{i}^{2}\\
    \end{align}
    El valor de $\lambda$ corresponde con cuanto se quieren penalizar los coeficientes de los pesos $w_{i}$.
    \item \textbf{\textit{Dropout}:} es una técnica aplicada a todas las neuronas de una capa, consiste en que a cada neurona de la capa en cuestión tendrá asociada una probabilidad de desactivación que puede estar entre 0 y 100\% \cite[p~276]{Szeliski2022}, por lo que la red tendría un funcionamiento como el de la Figura \ref{dropout}.
    \begin{figure}[H]
        \centering
        \includegraphics[scale=0.7]{Recursos/Dropout.jpg}
        \caption[Efecto del \textit{Dropout} en el entrenamiento]{Efecto del \textit{Dropout} en el entrenamiento. {\footnotesize Fuente: \textit{Computer Vision:
Algorithms and Applications
2nd Edition}\cite[p~276]{Szeliski2022}}}
        \label{dropout}
    \end{figure}
    Colocar neuronas aleatoriamente a cero inyecta ruido en el proceso de formación y también evita que la red se especialice demasiado  sus neuronas a muestras o tareas particulares, de tal forma que se reduce el sobreajuste y mejorar la generalización.
    \item \textbf{Aumento del conjunto de datos:} Otra técnica poderosa para reducir el ajuste excesivo es agregar más muestras de entrenamiento perturbando las entradas y / o salidas de las muestras que ya han sido recolectadas. Esta técnica es eficaz en tareas de clasificación de imágenes, ya que es caro obtener ejemplos etiquetados, y también dado que las clases de imágenes no deben cambiar bajo pequeñas perturbaciones \cite[p~275]{Szeliski2022}.
\end{itemize}
\subsubsection{Algoritmo de backpropagation}
Propuesto en 1986 por Rumelhart, Hinton y Williams, es el responsable de que una red neuronal sea capaz de auto ajustar sus parámetros para así aprender una representación interna de la información que estaba procesando. Para su ejecución se requiere de un algoritmo conocido como descenso al gradiente, el cual consiste en que en cada iteración del entrenamiento se evalúe el error del modelo y se calculen las derivadas parciales (gradientes) de dicho error, los gradientes indicaran la pendiente de la función hacia donde el error incrementa, por lo que para reducir el error en cada iteración, es necesario substraer el vector de gradientes al resultado final. 
\\
No obstante, debido a la estructura de una red neuronal, los valores de los parámetros en las capas posteriores dependen de las capas previas y a su vez de las conexiones entre neuronas, por lo que calcular el gradiente de forma directa no es una opción, la solución es utilizar el descenso del gradiente para optimizar la función de coste (función de error) empleando la técnica de backpropagation para calcular el vector de gradientes dentro de la complejidad de la arquitectura de la red. 
\\
El funcionamiento de este algoritmo parte de analizar desde el final de la red con la señal de error hacia las primeras capas, la razón del porqué se va desde la última capa hacia la primera capa, es que en una red neuronal el error de las capas anteriores depende directamente del error de las capas posteriores, teniendo en cuenta este hecho, es posible determinar cuál es el efecto de cada neurona en el resultado final mediante la retro propagación de errores, de este modo es posible computar cuanto hay que modificar cada parámetro en cada neurona. Además una vez aplicados los errores a las neuronas de la capa de turno se puede proceder a repetir el mismo proceso previo como si este fuera el error de la red; es decir, asumiendo que la capa de turno es la nueva última capa, así aplicar \textit{backpropagation} es operar siempre de forma recursiva capa tras capa moviendo el error hacia atrás. Entonces al alcanzar la primera capa se habrá obtenido cuál es el error para cada neurona y para cada uno de sus parámetros, dichos errores son usados para calcular las derivadas parciales de cada parámetro de la red, conformado así el vector de gradientes al cual se le aplica el descenso al gradiente para lograr minimizar el error \cite[p~284]{Szeliski2022}. 
\subsection{Redes neuronales convolucionales ó \textit{Convolutional Neural Networks} (CNN)}
Es una arquitectura de red empleada en el procesamiento de imágenes, cuya principal ventaja respecto a las redes estudiadas previamente, es su eficiencia al momento del entrenamiento, ya que la cantidad de parámetros a ajustar se reduce considerablemente. A diferencia de una red de capas FC las capas de las redes CNN consisten en un conjunto de filtros que se van ajustando a medida que avanza el aprendizaje, cada filtro suele ser pequeño espacialmente (tanto en ancho como en alto) respecto a las dimensiones de la imagen de entrada, pero estos se extienden a través de todas las dimensiones del volumen de entrada, se le conoce como volumen de entrada debido a que una imagen con los 3 canales (RGB) corresponde con un tensor de dimensiones $WxHxD$ (ancho x alto x profundidad). Al introducir una imagen en la red se desliza (más precisamente, se realiza la convolución) cada filtro a través de todo el ancho y alto del volumen de entrada y se calculan los productos escalares entre las entradas del filtro y la entrada en cualquier posición.  A medida que se desliza el filtro sobre el ancho y el alto del volumen de entrada, se producirá un mapa de activación bidimensional o mapa de características, que da las respuestas de ese filtro en cada posición espacial (Ver Figura \ref{fowardPassCNN}).
\begin{figure}[H]
    \centering
    \includegraphics[scale=0.7]{Recursos/fowardPassCNN.jpg}
    \caption[Convolución entre el volumen de entrada de dimensiones 7x7x3 y dos filtros de dimensiones 3x3x3.]{Convolución entre el volumen de entrada de dimensiones 7x7x3 y dos filtros de dimensiones 3x3x3. {\footnotesize Fuente: CS231n\textit{ Convolutional Neural Networks for Visual Recognition} \cite{CS231n}}}
    \label{fowardPassCNN}
\end{figure}
De forma intuitiva, la red aprenderá filtros que se activan cuando ven algún tipo de característica visual, como un borde de alguna orientación o una mancha de algún color en la primera capa, o eventualmente patrones enteros en forma de panal o rueda en capas superiores de la red. De este modo se tendrá un conjunto completo de filtros en cada capa, en lugar de neuronas y cada uno de ellos producirá un mapa de activación bidimensional separado. Se Apilaran estos mapas de activación a lo largo de la dimensión de profundidad generando así el volumen de salida\cite{CS231n}.
\\ 
En las CNN se suelen emplear dos tipos de capas, aquellas donde se realiza la convolución, las cuales fueron descritas previamente y las capas de agrupación o \textit{polling layers}, estas son insertadas periódicamente entre capas de convolución con el fin de reducir progresivamente el tamaño espacial de la representación para así disminuir la cantidad de parámetros y cálculos en la red y, por lo tanto, controlar también el \textit{overfitting}. La capa de agrupación funciona de forma independiente en cada segmento de profundidad de la entrada y la redimensiona espacialmente, utilizando la operación \textbf{MAX}. La forma más común es una capa de agrupación con filtros de tamaño 2x2 aplicados con un paso de 2, por lo que con estos parámetros la operación \textbf{MAX} tomaría el valor máximo entre 4 números en una región de 2x2 (Ver Figura \ref{maxPooling}) \cite{CS231n}. De manera más general, la capa de agrupación:
\begin{itemize}
    \item Acepta un volumen de dimensiones $W_{1}xH_{1}xD_{1}$
    \item Requiere dos hiperparámetros:
    \begin{itemize}
        \item Su extensión espacial ``F''
        \item El paso o zancada ``S''
    \end{itemize}
    \item Produce un volumen $W_{2}xH_{2}xD_{2}$.
    \item Introduce cero parámetros, ya que calcula una función fija de la entrada
    \item Para las capas de agrupación, no es común rellenar la entrada con relleno (\textit{padding}) de ceros, como en el caso de una capa de convolución.
\end{itemize}
\begin{figure}[H]
    \centering
    \includegraphics[scale=0.6]{Recursos/maxPolling.jpg}
    \caption[Capa de agrupación para un volumen de entrada de 224x224x64 con F = 2, P = 2 y un volumen de salida de 112x112x32.]{Capa de agrupación para un volumen de entrada de 224x224x64 con F = 2, P = 2 y un volumen de salida de 112x112x32. {\footnotesize Fuente: CS231n\textit{ Convolutional Neural Networks for Visual Recognition} \cite{CS231n}}}
    \label{maxPooling}
\end{figure}
\subsection{Hiperparámetros de las redes CNN}
Son las variables que rigen el proceso de entrenamiento en sí, es decir; son variables de configuración. En el caso de las capas de convolución se tienen:
\begin{enumerate}
    \item \textbf{La profundidad del volumen de salida:} corresponde a la cantidad de filtros a usar en cada capa y se identifica por $D_{n}$ donde $n$ es el número de la capa en cuestión. Sin embargo, también puede identificarse mediante $K$.
    \item \textbf{El campo receptivo de la neurona:} es equivalente al tamaño del filtro y se identifica mediante $F$.
    \item \textbf{El paso o zancada:} corresponde con la cantidad de píxeles por las que se desliza el filtro al momento de realizar la convolución, se identifica mediante la letra $S$. Por ejemplo cuando el paso es 1, los filtros se desplazan un píxel a la vez. En el caso de ser 2 los filtros saltan 2 píxeles a la vez a medida que se deslizan por el volumen de entrada, de esta forma se producirán volúmenes de salida más pequeños espacialmente.
    \item \textbf{El relleno ó \textit{padding}:} hay casos donde será conveniente rellenar el volumen de entrada con ceros alrededor del borde, como es el caso de la Figura \ref{fowardPassCNN}. Por lo tanto el tamaño del relleno de ceros es un hiperparámetro que  permitirá controlar el tamaño espacial de los volúmenes de salida, aunque su aplicación más común es la de preservar exactamente el tamaño espacial del volumen de entrada para que el ancho de entrada y salida y la altura sean los mismos.
    \item \textbf{ Dilatación:} este parámetro permite tener filtros que tengan espacios entre cada celda, por ejemplo; en el caso de una dimensión un filtro $w$ de tamaño 3 calcularía sobre la entrada $x$ lo siguiente: $w [0] * x [0] + w [1] * x [1] + w [2] * x [2]$. Este es en el caso de que no exista dilatación (dilatación = 0). Para la dilatación = 1, el filtro calcularía $w [0] * x [0] + w [1] * x [2] + w [2] * x [4]$; En otras palabras, hay una brecha de 1 entre las aplicaciones.  Esto puede ser muy útil en algunas configuraciones para usar junto con filtros de dilatación 0 porque le permite fusionar información espacial a través de las entradas de manera mucho más agresiva con menos capas.
\end{enumerate}
De acuerdo con los hiperparámetros, una capa convolucional se caracteriza por:
\begin{itemize}
    \item Acepta un volumen de dimensiones $W_{1}xH_{1}xD_{1}$.
    \item Requiere 4 hiperparámetros:
    \begin{itemize}
        \item El numero de filtros ``K''
        \item La extensión de los filtros ``F''
        \item El paso ``S''
        \item la cantidad de relleno ``P''.
    \end{itemize}
    \item Produce un volumen $W_{2}xH_{2}xD_{2}$ donde:
    \begin{itemize}
        \item $W_{2}$ $=$ $(W_{1} - F + 2P)/S+1$
        \item $H_{2}$ $=$ $(H_{1} - F + 2P)/S+1$
        \item $D_{2}$ $=$ $K$
    \end{itemize}
\end{itemize}
\section{Detección de objetos usando YOLO}
Entre las formas de reconocimiento existentes basadas en redes neuronales \textit{You Only Look Once} (YOLO) es una arquitectura de red capaz de conseguir una velocidad de predicción de 45 cuadros por segundo en su forma base y 155 cuadros con \textit{fast} YOLO. Por otro lado, una cámara de teléfono captura vídeos alrededor de 30 cuadros por segundo, mientras que las cámaras de alta velocidad alcanzan alrededor de 250 cuadros por segundo. La velocidad de detección en milisegundos de YOLO está entre 6 a 22 ms, a comparación con el cerebro humano que es capaz de detectar alrededor de 13 ms. A este tipo de arquitectura se le introducen imágenes y entrega como resultado \textit{bounding boxes} o cuadros delimitadores que engloban al objeto en cuestión y la clase a la que pertenece dicho objeto \cite[p~188]{Krishnendu}.
\\
\\
YOLO utiliza un elemento llamado intersección sobre la unión (IoU) para poder evaluar el grado de solapamiento entre una predicción y el \textit{ground truth} (salidas esperadas etiquetadas a mano por seres humanos), dicha métrica está dada por la siguiente expresión: 
\begin{equation}
    IoU = \frac{\text{Área de solapamiento}}{\text{Área de unión}} = \frac{\text{Área común entre bounding boxes}}{\text{Área total cubierta por los bounding boxes}}
\end{equation}
En la Figura \ref{IoU} se puede apreciar un ejemplo de una IoU cercana a 0.9, debido a que el grado de solapamiento es bastante alto, en caso de que los dos \textit{bounding boxes} no se solapen el IoU sería de 0, de no ser así dicha métrica alcanza un valor de 1. Esta es la forma de evaluar que tan buena es una predicción.
\begin{figure}[H]
    \centering
    \includegraphics{Recursos/iou.png}
    \caption[IoU: Predicción vs. \textit{ground truth}.]{IoU: Predicción vs. \textit{ground truth}. {\footnotesize Fuente: \textit{Mastering Computer Vision with TensorFlow 2.x} \cite[p~189]{Krishnendu}}}
    \label{IoU}
\end{figure}
\subsection{Arquitectura de YOLOv3}
La versión número 3 de esta red tiene como base una CNN, que simultáneamente predice las coordenadas de múltiples \textit{bounding boxes} y la probabilidad de detectar un objeto dado en cada \textit{bounding box}.
\begin{figure}[H]
    \centering
    \includegraphics[scale=0.8]{Recursos/yolov3_architecture.png}
    \caption[Arquitectura de red YOLOv3.]{Arquitectura de red YOLOv3. {\footnotesize Fuente: \textit{Mastering Computer Vision with TensorFlow 2.x} \cite[p~192]{Krishnendu}}}
    \label{yolov3Architecture}
\end{figure}
 Esta versión del detector está compuesta por 24 capas convolucionales y dos FC (ver Figura \ref{yolov3Architecture}). Su mecanismo de detección es realizado a 3 diferentes escalas, específicamente en las capas 82, 94 y 106. La red utiliza 23 capas convolucionales y bloques residuales como los de la Figura \ref{tipos_de_bloques_residuales} entre las capas 1 y 74, donde el tamaño de la imagen de entrada cae de 608 a 19 píxeles y la profundidad aumenta de 3 a 1024 a través de los filtros alternos 3 x 3 y 1 x 1. El \textit{by-pass} o salto de conexión característico de los bloques residuales ayuda a reducir el problema del desvanecimiento de los pesos al que se enfrentan las arquitecturas de redes profundas, lo que se traduce en que el algoritmo del descenso del gradiente pueda alcanzar su óptimo.
 \begin{figure}[H]
     \centering
     \includegraphics[scale=0.6]{Recursos/residual_block.png}
     \caption[Tipos de bloques residuales.]{Tipos de bloques residuales. {\footnotesize Fuente: \textit{Identity Mappings in Deep Residual Networks} \cite{residualBlocksPaper}}}
     \label{tipos_de_bloques_residuales}
 \end{figure}
Cada bloque residual empleado en esta arquitectura es sucedido por un bloque de pre-convolución de dimensiones 1 x 1 y filtros de dimensiones 3 x 3 hasta la primera detección en la capa 82. En la Figura \ref{yolov3Architecture} se observan dos saltos o \textit{by-passes}, el primero entre las capas 61 a la 85 y el segundo entre la 36 a la 97. Así mismo, el paso o zancada se mantiene como 1 la mayor parte del tiempo, excepto en 5 casos, donde un valor de zancada de
2 se utiliza para la reducción de tamaño, junto con un filtro de 3 x 3.
\subsection{Funcionamiento}
A continuación se describirán de forma detallada los pasos que lleva a cabo esta arquitectura, para predecir los \textit{bounding boxes} y la clase de uno o varios objetos en una imagen.
\begin{enumerate}
    \item La CNN en YOLO usa las características (\textit{features}) de la imagen completa para predecir cada \textit{bounding box}, por este motivo la predicción es global, en lugar de ser local.
    \item La imagen es dividida en celdas de una cuadrícula $S$ $x$ $S$, donde $S$ es una cantidad finita. Cada celda predice $B$ \textit{bounding boxes} con una probabilidad ($P$). Entonces el número total de \textit{bounding boxes} estará dado por $S$$x$$S$$x$$B$, con su correspondiente probabilidad para cada \textit{bounding box}.
    \item Cada \textit{bounding box} posee un vector de predicciones tal que, (x, y, w, h, c) el cual puede ser desglosado de la siguiente forma:
    \begin{itemize}
        \item (x, y): son las coordenadas del centro del \textit{bounding box}, respecto a la coordenada de la celda ubicada en la cuadrícula.
        \item (w, h): son las dimensiones de ancho y alto respectivamente del \textit{bounding box} en cuestión, respecto a las dimensiones de la imagen
        \item c: es el grado de confianza el cual representa el IoU entre la predicción y el \textit{ground truth}.
    \end{itemize}
    \item La probabilidad de que una celda en la cuadrícula contenga un objeto, es definida por la probabilidad de que pertenezca a una clase dada multiplicada por el IoU, esto implica que si una celda contiene parcialmente a un objeto su probabilidad será baja y su valor de IoU se mantendrá bajo. Este hecho produce dos efectos en el \textit{bounding box} de esa celda:
    \begin{itemize}
        \item La forma del \textit{bounding box} será más pequeña que el tamaño del \textit{bounding box} de la celda que incluye completamente el objeto porque la celda solo puede ver parte del objeto e infiere su forma a partir de ello. Además, si la celda contiene una parte muy pequeña del objeto no reconocerá el objeto del todo.
        \item El grado de confianza en la clase será pequeño debido a que el valor del IoU resultante de una imagen parcial no se ajustara con la predicción del \textit{ground truth}.
    \end{itemize}
    \item En general, una celda puede contener solo una clase, pero usando el principio del \textbf{\textit{Anchor box}} múltiples clases pueden ser asignadas a una celda. Un \textit{anchor box} como se puede ver en la Figura \ref{anchor_boxes} son regiones cuyas dimensiones son predefinidas por una expresión matemática o de acuerdo a la forma de la clase detectada. Por ejemplo: si se detectan tres clases como carro, motocicleta y humano, entonces es probable que utilizando dos \textit{anchor boxes}, uno para humano y motocicleta y otro para el carro se logre la predicción deseada, esto se puede confirmar observando como en la Figura \ref{exampleYOLO} donde el objeto más a la izquierda corresponde con un carro. Para ajustar la forma de un \textit{anchor box} sin emplear una ecuación matemática generadora se puede analizar la forma de cada clase en el conjunto de entrenamiento de la red usando algoritmos como \textit{k-means clustering}\cite[p~191]{Krishnendu}.
    \begin{figure}[H]
        \centering
        \includegraphics[scale=0.7]{Recursos/anchor_boxes.png}
        \caption[Anchor boxes para una región con relaciones de aspecto [1, 2, $\frac{1}{2}$].]{Anchor boxes para una región con relaciones de aspecto [1, 2, $\frac{1}{2}$]. {\footnotesize Fuente: \textit{Advanced Deep Learning with Keras} \cite[p.~376]{Atienza2018}}}
2        \label{anchor_boxes}
    \end{figure}
\end{enumerate}
En la Figura \ref{exampleYOLO} se tienen las tres clases ya mencionadas carro, motocicleta y humano asumiendo una cuadrícula de 5 x 5 con 2 \textit{anchor boxes} y 8 dimensiones 5 para los parámetros del \textit{bounding box} (x, y, w, h, c) y 3 para las clases (c1, c2, c3). Por lo tanto el vector de salida es 5 x 5 x 2 x 8.
\begin{figure}[H]
    \centering
    \includegraphics{Recursos/grid_example.png}
    \caption[Cálculo de las coordenadas de la predicción. Nota: el tamaño de la imagen es 448 x 448 (solo con propósitos de ilustración), se muestra el método de cómputo de las clases humano y carro y cada \textit{anchor box} tiene un tamaño que puede ir desde 448/5 hasta 89 píxeles.]{Cálculo de las coordenadas de la predicción. Nota: el tamaño de la imagen es 448 x 448 (solo con propósitos de ilustración), se muestra el método de cómputo de las clases humano y carro y cada \textit{anchor box} tiene un tamaño que puede ir desde 448/5 hasta 89 píxeles. {\footnotesize Fuente: \textit{Mastering Computer Vision with TensorFlow 2.x} \cite[p~191]{Krishnendu}}}
    \label{exampleYOLO}
\end{figure}
\section{Segmentación}
Es un método de reconocimiento enfocado en clasificar cada píxel de una imagen de acuerdo a una clase correspondiente. Los algoritmos de segmentación fragmentan una imagen en conjuntos de píxeles o regiones con el propósito de tener un mejor entendimiento de lo que representa la imagen. De este modo el campo de la segmentación cambia dependiendo de la aplicación, los 3 casos más comunes son los siguientes:
\begin{itemize}
    \item \textbf{Segmentación de instancias:} se utiliza cuando el punto de interés son objetos contables dentro de una imagen, por lo tanto el resto de los objetos que no son de interés son clasificados como fondo o \textit{background}\cite{Atienza2018}, tal es el caso de la Figura \ref{instance_segmentation}.
    \begin{figure}[H]
        \centering
        \includegraphics[scale=0.5]{Recursos/instance_segmentation.png}
        \caption[Ejemplo de segmentación de instancias.]{Ejemplo de segmentación de instancias. {\footnotesize Fuente: \textit{Advanced Deep Learning with Keras} \cite{Atienza2018}}}
        \label{instance_segmentation}
    \end{figure}
    \item \textbf{Segmentación semántica:} cuando el centro de atención no son objetos contables, sino más bien regiones amorfas difíciles de contar como el cielo, bosques, vegetación, caminos, grama, edificios, etc \cite{Atienza2018}.
    \item \textbf{Segmentación panóptica:} cuando se clasifican tanto los objetos contables como no contables es decir la imagen completa \cite{Atienza2018}.
\end{itemize}
Como se pudo observar existen una gran variedad de técnicas cuando de segmentación se habla, aunque entre ellas la segmentación semántica se hizo bastante popular debido al modelo de código abierto desarrollado por Google DeepLab y su amplio campo de aplicación.
\subsection{Segmentación semántica con DeepLab}
En general posee una arquitectura basada en un \textit{encoder-decoder}, mientras que el \textit{encoder} toma la imagen de entrada (Ver Figura \ref{encoder_decoder}) para generar un vector de características, el \textit{decoder} realiza el procedimiento inverso generando una imagen a partir de dicho vector \cite[p~220]{Krishnendu}. 
\begin{figure}[H]
    \centering
    \includegraphics[scale=0.6]{Recursos/encoder_decoder.png}
    \caption[Arquitectura de DeepLab.]{Arquitectura de DeepLab. {\footnotesize Fuente: \textit{Fully Convolutional Networks for Semantic Segmentation} \cite{fullyConectedNet}}}
    \label{encoder_decoder}
\end{figure}
Las características claves del \textit{encoder} empleado son:
\begin{itemize}
    \item Emplea convoluciones dilatadas para extraer las características en las imágenes de entrada, estas como se puede ver en la Figura \ref{dilated_convolution} incrementan el campo de visión de la convolución. Mientras que una convolución normal emplea la zancada y capas de agrupación para reducir las dimensiones de salida de las capas, reduciendo la así la resolución de los mapas de características, lo cual es negativo para la segmentación, este método soluciona dicho problema. 
    \begin{figure}[H]
        \centering
        \includegraphics[scale=0.6]{Recursos/atrous_convolution.png}
        \caption[Comparación entre convolución común (izquierda) y convolución dilatada (derecha).]{Comparación entre convolución común (izquierda) y convolución dilatada (derecha). {\footnotesize Fuente: \textit{Mastering Computer Vision with TensorFlow 2.x} \cite[p~219]{Krishnendu}}}
        \label{dilated_convolution}
    \end{figure}
    \item El paso o zancada de salida esta dado por la relación entre la resolución de la imagen de entrada y la salida final. Un valor típico para esto es 16 u 8, lo que resulta en \textit{features} más densas.
    \item Un modulo ASPP (\textit{atrous spatial pyramid polling}) usado para realizar la convolución a diferentes escalas. Compuesto por convoluciones dilatadas apiladas en paralelo una detrás de otra para formar una \textit{spatial pyramid polling} (SPP).  
\end{itemize}
Por otro lado el \textit{decoder} posee las siguientes características:
\begin{itemize}
    \item Utiliza una convolución de dimensiones 1 x 1 para reducir los canales de los mapas de características de bajo nivel (producidos por el \textit{encoder}).
    \item Emplea convoluciones de dimensiones 3 x 3 para obtener una segmentación más nítida.
    \item El factor de re-escalado es de 4, es decir; las dimensiones de ancho y alto de una capa a otra se cuadruplican hasta alcanzar las dimensiones originales.
\end{itemize}
Al implementar esta arquitectura es común que el \textit{encoder} sea una red pre-entrenada de clasificación como lo son VGG o ResNet, utilizar redes pre-entrenadas para reentrenarlas con otros propósitos es llamado \textbf{Transferencia de Aprendizaje}. Al aplicar la transferencia el \textit{decoder} puede concentrarse en proyectar las \textit{features} de menor resolución producidas por el \textit{encoder} en un espacio de mayor resolución, para así obtener una clasificación densa de píxeles y a su vez reducir el tiempo de entrenamiento considerablemente.
\section{El paquete de \textit{Python} y los bloques que lo conforman}
\textit{Python} es un lenguaje de código abierto, con una comunidad activa de colaboradores y usuarios que comparten su software para que otros desarrolladores lo utilicen. Son colaboradores, aquellos que comparten sus códigos o \textit{scripts}, en forma de paquetes en el \textit{Python Packaging Index} (PyPI). Los paquetes son similares a directorios de un sistema de archivos, ya que estos permiten separar el código de forma jerárquica \cite{Paquetes}. 
\subsection{Módulos de un paquete}
Se conoce como modulo a todo fichero o \textit{script} de \textit{Python} cuya extensión es .py. Separar el código en módulos permite facilitar el mantenimiento del programa a medida que crezca. Algunos aspectos a tomar en cuenta sobre los módulos serían:
\begin{itemize}
    \item Pueden contener tanto declaraciones ejecutables como definiciones de funciones. Estas declaraciones están pensadas para inicializar el módulo. Se ejecutan únicamente la primera vez que el módulo se encuentra en una declaración \textit{import}\cite{ModuloPython}.
    \item Cada módulo tiene su propio espacio de nombres, el cual es usado como espacio de nombres global para todas las funciones definidas en su interior.
\end{itemize}
\subsection{Tipos de \textit{scripts} de configuración del paquete}
A continuación se presentarán los dos paquetes utilizados en \textit{Python} para la distribución de código en el PyPI:
\subsubsection{Distutils}
Es un paquete de \textit{Python} que da soporte, para la creación, distribución e instalación de módulos, los módulos pueden ser hechos únicamente en \textit{Python} o escritos en el lenguaje C. Distutils como paquete presenta las siguientes funcionalidades:
\begin{itemize}
    \item Soporte para declarar dependencias del proyecto, siendo las dependencias aquellos paquetes o módulos externos necesarios para el funcionamiento del proyecto.
    \item Mecanismos adicionales para configurar cuáles archivos incluir en lanzamientos de código fuente, esto se logra mediante el uso de sistemas de control de versiones.
    \item La capacidad de generar, automáticamente, ejecutables de línea de comandos de \textit{Windows} en el momento de la instalación, en lugar de tener que compilarlos previamente.
    \item Comportamiento consistente en todas las versiones de \textit{Python} soportadas por el paquete.
\end{itemize}
Un ejemplo de como estructurar el archivo setup.py que maneja Distutils sería:
\begin{figure}[H]
    \begin{minted}{Python}
		from distutils.core import setup
		setup(name='Distutils',
                version='1.0',
                description='Python Distribution Utilities',
                author='Greg Ward',
                author_email='gward@python.net',
                url='https://www.python.org/sigs/distutils-sig/',
                packages=['distutils', 'distutils.command'],)
	\end{minted}
    \caption[Archivo de configuración setup.py utilizando el paquete Distutils.]{Archivo de configuración setup.py utilizando el paquete Distutils. {\footnotesize Fuente: \textit{Writing the Setup Script} \cite{python-software-foundation}}}
    \label{ejDistutils}
\end{figure}
En la Figura \ref{ejDistutils} es importante resaltar la opción \textit{packages} debido a que esta le dice a Distutils que procese (compile, distribuya, instale, etc.) todos los módulos de \textit{Python} que se encuentran en cada paquete mencionado en la lista de \textit{packages}.
\subsubsection{Setuptools}
Es un paquete que contiene una serie de mejoras a las capacidades de Distutils, cuya cualidad principal es permitir que los desarrolladores puedan distribuir paquetes de forma más sencilla, con especial énfasis en paquetes que posean dependencias externas. Esto no implica que distutils sea obsoleto, puesto que setuptools aún se encuentra en desarrollo. A diferencia de su contra parte, las configuraciones requieren de al menos 2 archivos y un paquete, el archivo pyproject.toml; es aquel donde se debe declarar que se empleara Setuptools, en la Figura \ref{pyproject.toml} se puede observar la estructura de dicho archivo.
\begin{figure}[H]
    \centering
    \begin{minted}{Python}
		[build-system]
        requires = ["setuptools", "wheel"]
        build-backend = "setuptools.build_meta"
	\end{minted}
    \caption[Archivo pyproject.toml.]{Archivo pyproject.toml. {\footnotesize Fuente: \textit{Build System support} \cite{python-packaging-authority}}}
    \label{pyproject.toml}
\end{figure}
El archivo setup.cfg es el reemplazo al archivo setup.py y puede conformarse de la siguiente manera:
\begin{figure}[H]
    \centering
    \begin{minted}{Python}
		[metadata]
        name = "mypackage"
        version = 0.0.1

        [options]
        packages = "mypackage"
        install_requires =8
        requests
        importlib; python_version == "2.6"
        
        [options.entry_points]
        console_scripts =
        main = mypkg:some_func
	\end{minted}
    \caption[Archivo setup.cfg.]{Archivo setup.cfg. {\footnotesize Fuente: \textit{Build System support} \cite{python-packaging-authority}}}
    \label{setup.cfg}
\end{figure}
Como se puede ver en la Figura \ref{setup.cfg} el archivo es bastante similar en estructura al fichero setup.py; no obstante, tiene varias ventajas, ya que los elementos que pertenecen a la opción \mintinline{Python}{install_requires}, son directamente las dependencias del paquete, e incluso al momento de instalar un paquete externo, setuptools actualiza las dependencias necesarias. Para proyectos muy grandes es posible automatizar la búsqueda de paquetes, con el fin de que no sea necesario declarar cada paquete en el fichero setup.cfg. 
Las herramientas de configuración admiten la creación automática de scripts tras la instalación mediante "\mintinline{Python}{options.entry_points}". Un paquete que emplee setuptools debe poseer una estructura de directorios como en la Figura \ref{estructuraSetupTools}.
\begin{figure}[H]
    \centering
    \includegraphics{Recursos/estructuraSetupTools.jpg}
    \caption[Estructura de paquete mediante setuptools.]{Estructura de paquete mediante setuptools. Fuente: El Autor}
    \label{estructuraSetupTools}
\end{figure}
El último elemento que requiere Setuptools es el paquete pep517, el cual se puede instalar mediante el instalador de paquetes de \textit{python} "pip", posteriormente se debe invocar con el comando \mintinline{Python}{python -m pep517.build}.
\subsection{El paquete wheel}
Es un proyecto compatible con distutils y setuptools, que produce un formato de empaquetado binario multiplataforma (llamado ``wheels'' o ``wheel \textit{files}'' y definido en PEP 427) es decir; la función de wheels es permitir la distribución de paquetes sin importar el sistema operativo empleado \cite{wheel}.
\subsection{Pruebas de regresión mediante módulo unittest}
El objetivo de las pruebas de regresión implementadas mediante el módulo unittest, es probar situaciones donde el código deje de funcionar, para encontrar errores en el mismo y solventar problemas en el software, de acuerdo con la documentación oficial \cite{pruebas-regresion} existen una serie de pautas a seguir para ejecutar dichas pruebas, las más importantes en el caso del software a desarrollar son:
\begin{itemize}
    \item El conjunto de pruebas se debe hacer a todas las clases, funciones y constantes.
    \item Importar la menor cantidad de módulos posible. Esto minimiza las dependencias externas de las pruebas y también minimiza el posible comportamiento anómalo de los efectos secundarios de importar un módulo.
    \item Maximizar la reutilización del código. En ocasiones, las pruebas variarán en algo tan pequeño como qué tipo de entrada se utiliza.
    \item Asegurar la limpieza después de las pruebas (así como cerrar y eliminar todos los archivos temporales).
    \item Asegurar que todos los valores posibles son probados, incluidos los no válidos. Esto permite que no solo todos los valores válidos sean aceptables, sino que los valores incorrectos se manejen correctamente.
    \item Añadir una prueba explícita para cualquier error descubierto para el código probado. Esto asegurará que el error no vuelva a aparecer si el código se cambia en el futuro.
\end{itemize}%

%==================================================================
\chapter{DESCRIPCIÓN DEL SISTEMA DE ADQUISICIÓN DE DATOS}\label{CAP:hardware}
%\markboth{Tu Segundo Capítulo}{Tu Segundo Capítulo}%
\section{Diseño del soporte físico para el sistema de adquisición de data}
En el caso de un sistema estereo se entiende por sistema de adquisicion de datos al conjunto compuesto por ambas camaras. De acuerdo con lo mencionado en el capitulo II, para seleccionar y ajustar dicho sistema es necesario tomar en cuenta: 
\section{Diseño de la arquitectura del paquete}
El paquete sera diseñado sobre el lenguaje de programación python y de ser necesario se utilizaran segmentos de códigos hechos en c++ para tareas que requieran un alto rendimiento, además de esto se emplearan las siguientes dependencias para agilizar y optimizar el desarrollo:
\begin{itemize}
    \item Numpy, Scipy, matplotlib, y pandas: para el manejo algebraico y presentación de los datos.
    \item OpenCV: Para recibir la entrada de datos y exportar en formatos de video o imagen sea el caso.
    \item Tensorflow/keras: como framework de machine learning.
\end{itemize}
\subsection{Requisitos del paquete}
\begin{itemize}
    \item El paquete debe ser agnostico en cuanto al tipo de entrada; es decir, debe ser capaz de aceptar 2 imagenes RGB o imagenes en blanco y negro o videos que pueden ser previamente grabados o  captados en el momento.
    \item Debe poseer un modulo que permita realizar varias técnicas de preprocesamiento para mejorar la calidad de las imágenes captadas, esto en pro de mejorar la inferencia de los modelos.
    \item Por medio de un API (Application program Interface) debe ser capaz de construir redes neuronales con diferentes arquitecturas basadas en redes ya empleadas para las tareas de reconocimiento y calculo del mapa de disparidad.
    \item El modulo o los modulos de reconocimiento deben estar basados en modelos pre-entrenados, puesto que su papel es reconocer objetos en una escena unicamente. De modo que los modulos de calculo de profundidad tendran una mayor personalización y requeriran entrenamiento.
    \item Debe existir un modulo que realice una rapida implementación del paquete utilizando al menos dos de las metodologias de aprendizaje automatico estudiadas.
    \item Debe existir un modulo que permita cargar dataset empleados en vision estereo como lo son: KITTI, Middlebury, .....
    \item Debe existir un modulo compuesto por varios submodulos que separen el proceso de entrenamiento de una red para el calculo del mapa de disparidad, de acuerdo con las etapas descritas en el capitulo II: extracción de características, correspondencia de caracteristicas, regularizacion del costo y refinamiento.
    \item Debe existir un modulo que active o desactive la posibilidad de usar la GPU en el entrenamiento
    \item Debe existir un modelo que permita cargar modelos pre-entrenados con el fin de probar infraestructuras conocidas.
    \item Por ultimo el paquete debe ser capaz de entregar como salida la inferencia de los módulos de reconocimiento o los mapas de disparidad finales o la posición en el espacio de un objeto o los objetos detectados por el modulo de reconocimiento.
    \item El paquete debe contar con un banco de pruebas unitarias y se capaz de autodomentarse a medida que se realiza el codigo.
\end{itemize}%

%==================================================================
\chapter{DESCRIPCIÓN DEL ENTORNO DE TRABAJO DEL PAQUETE}\label{CAP:entorno}
%\markboth{Tu Segundo Capítulo}{Tu Segundo Capítulo}%
  \input{ch4.tex}
%==================================================================  
\chapter{IMPLEMENTACIÓN DEL SOFTWARE}\label{CAP:software}
%\markboth{Tu Segundo Capítulo}{Tu Segundo Capítulo}%
  En este capítulo se describirá el funcionamiento del algoritmo estéreo, así como también se compararán algunos métodos de reconocimiento basados en redes neuronales, para posteriormente seleccionar uno y profundizar en la forma en la que se implementó. Por último, se presentará la manera en la que detección y posicionamiento son ejecutadas mediante el paquete. 
\section{El algoritmo de visión estéreo}
En el paquete desarrollado no es obligatorio ejecutar los pasos de post-procesamiento y preprocesamiento (ver Figura \ref{stereo_pipeline}) para obtener una salida estéreo, ambos procesos se ejecutan de igual forma para cualquiera de las 4 salidas posibles, no obstante es recomendable aplicarlos si se busca obtener un mejor mapa de disparidad. Actualmente el paso de correspondencia se realiza solo con el algoritmo de correspondencia de bloques semi globales (SGBM).
\begin{figure}[H]
    \centering
    \includegraphics[scale=0.4]{Recursos/stereo_pipeline.png}
    \caption{Diagrama de flujo del mecanismo estéreo}
    \label{stereo_pipeline}
\end{figure}
\subsection{Preprocesamiento}
Consiste en efectuar las siguientes acciones:
\begin{enumerate}
    \item Se rectifican las imágenes de entrada partiendo de cámaras previamente calibradas, es decir; que de antemano se conocen los parámetros intrínsecos e extrínsecos y las matrices de translación y rotación del sistema en su conjunto. Rectificar implica calcular nuevas matrices de rotación para cada cámara de tal forma que cada plano de imagen se encuentre en una disposición paralela virtualmente hablando, lo que provoca que las líneas epipolares sean paralelas simplificando así el problema de correspondencia a una dimensión.
    \\
    \\
    En el caso de la función empleada se distinguen dos casos:
    \begin{itemize}
        \item Cuando la disposición física de las cámaras posee únicamente un desplazamiento respecto al eje de las x, por lo que las matrices de proyección (P1 y P2) poseen la siguiente forma:
        \begin{align}
           P1 = \begin{bmatrix}
            f & 0 & c_{x1} & 0\\
            0 & f & c_{y} & 0\\
            0 & 0 & 1 & 0
            \end{bmatrix}\\
            P2 = \begin{bmatrix}
             f & 0 & c_{x2} & T_{x} \cdot f\\
            0 & f & c_{y} & 0\\
            0 & 0 & 1 & 0
            \end{bmatrix}
        \end{align}
        donde $c_{x1}$ y $c_{x2}$ son los centros de proyección en $x$ de cada cámara, $c_{y}$ es el centro de proyección en $y$, $f$ es la distancia focal y $T_{x}$ es la distancia en el eje x entre las cámaras.
        \item Cuando la disposición física de las cámaras posee únicamente un desplazamiento respecto al eje de las y, por lo que las matrices de proyección (P1 y P2) serian de la siguiente forma:
        \begin{align}
           P1 = \begin{bmatrix}
            f & 0 & c_{x} & 0\\
            0 & f & c_{y1} & 0\\
            0 & 0 & 1 & 0
            \end{bmatrix}\\
            P2 = \begin{bmatrix}
             f & 0 & c_{x} & 0\\
            0 & f & c_{y2} & T_{y} \cdot f\\
            0 & 0 & 1 & 0
            \end{bmatrix}
        \end{align}
        donde $T_{y}$ es la distancia vertical entre cámaras.
    \end{itemize}
    \item Se calculan los mapas de transformación que eliminan la distorsión causada por las irregularidades del lente y a su vez guardan la configuración rectificada las imágenes de entrada.
    \item Se le aplica una convolución a cada imagen con un filtro o kernel de la siguiente forma:
    \begin{align}
        \frac{1}{256}\begin{bmatrix}
            1 & 4 & 6 & 4 & 1\\
            4 & 16 & 24 & 16 & 4\\
            6 & 24 & 36 & 24 & 6\\
            4 & 16 & 24 & 16 & 4\\
            1 & 4 & 6 & 4 & 1\\
            \end{bmatrix}
    \end{align}
   Este kernel conocido como filtro Gaussiano reduce el ancho de banda de la imagen eliminando el ruido de alta frecuencia, para posteriormente eliminar todas las filas y columnas pares de la imagen lo que provoca que la nueva imagen tenga $1/4$ de la dimensión original.
\end{enumerate}
\subsection{Correspondencia de bloques semi globales (SGBM)}
A continuacion se listan los pasos básicos de este algoritmo \cite{LearningOpenCV3}:
\begin{enumerate}
    \item Se le aplica un filtro que normaliza el brillo de la imagen para reducir las diferencias de iluminación y mejorar la textura, luego se emplea un filtro de Sobel a toda la imagen, el cual es un operador diferencial discreto que calcula una aproximación al gradiente de la función de intensidad de una imagen, esto implica que dicho filtro realza los bordes existentes.
    \item Se pre-calcula un mapa de costos de pixel C(x, y, d) que coincida con las imágenes
izquierda y derecha utilizando las métricas de Birchfield-Tomasi.
    \item Se inicializa un mapa de costos 3D S(x, y, d) con ceros.
    \item Para cada una de las tres, cuatro, cinco u ocho direcciones (r) (ver Figura \ref{cost_directions}) se calcula $S^{r}(x, y, d)$. Para minimizar el uso de memoria las primeras tres o cinco direcciones (oeste, este, norte noroeste, noreste) se
    procesan juntas, y en el caso de ocho direcciones hay un segundo paso que procesa
    las tres direcciones restantes (sur, suroeste, sureste). En el caso del algoritmo de
    tres o cinco direcciones C(x, y, d) y S(x, y, d) no se almacenan explícitamente para
    todos los píxeles; solo se almacenan las últimas tres o cuatro filas de los búfers a
    la vez.
    \begin{figure}[H]
        \centering
        \includegraphics[scale=0.5]{Recursos/cost_directions.jpg}
        \caption{Caminos posibles para el cálculo de costos, Imagen de \cite{LearningOpenCV3}}
        \label{cost_directions}
    \end{figure}
    \item Una vez que S(x, y, d) está completo, se busca el valor mínimo y este es d*(x, y). Se utiliza una verificación de unicidad que calcula la diferencia entre la mejor coincidencia y la segunda mejor que se requiere para que la disparidad sea inequívoca, además se interpola el resultado para obtener una mejor respuesta
    \item Se realiza una verificación de izquierda a derecha para asegurar que las correspondencias sean consistentes y se marcan los píxeles sin coincidencias perfectas como ``disparidad no válida''.
    \item El SGBM puede tener problemas cerca de los límites de los objetos, ya que la ventana de coincidencia capta el primer plano de un lado y el fondo del otro lado, esto da como resultado una región local de grandes y pequeñas disparidades que se llaman moteado. Para evitar estas coincidencias límite, se establece un detector de manchas en una ventana de manchas, donde cada píxel se utiliza como base para la construcción de un componente conectado definido por un relleno de inundación de rango variable. El relleno de inundación de rango variable incluye un píxel vecino solo si está dentro de algún rango del píxel actual. Una vez que se calcula ese componente conectado, si es más pequeño que la ventana moteada, se considera moteado.
\end{enumerate}
El proceso anteriormente descrito requiere del ajuste de los siguientes parámetros:
\begin{itemize}
    \item \textbf{Disparidad mínima:} corresponde con el valor mínimo de disparidad buscado, comúnmente este valor es 0, pero dado que en ocasiones cuando se rectifica la imagen es desplazada hacia la izquierda o la derecha, este parámetro permite reajustar la posición de dicha imagen.
    \item \textbf{Número de disparidades:} fija el rango de las disparidades buscadas, esta ira desde el valor de la disparidad mínima sumando a este el número de disparidades, es decir; a mayor sea este valor la función de costos evaluara una mayor cantidad de píxeles, lo que implica que si se tiene un sistema con una línea base grande entre cámaras en una disposición alineada este parámetro incrementara en consecuencia.
    \item \textbf{Tamaño de bloque:} índica el tamaño de la ventana deslizante utilizada en el cálculo de coincidencias.
    \item \textbf{Disp12MaxDiff:} Dado que este algoritmo realiza el cálculo de correspondencias de la imagen izquierda a la derecha y viceversa, este valor define la diferencia mínima entre ambas correspondencias.
    \item \textbf{Rango de motas:} las motas son producidas cerca de los bordes de los objetos debido a que la ventana de correspondencia captura el plano del objeto de un lado y el fondo del otro, por lo que para eliminar estos artefactos se le aplica un filtro de motas, al cual a través de este parámetro es posible controlar que tan cerca deben encontrarse los valores de disparidad para considerarse parte del mismo bloque.
    \item \textbf{Tamaño de la ventana moteada:} es el número de píxeles por debajo del cual una mancha de disparidad se descarta como ``moteada''. 
    \item \textbf{P1 y P2:} son dos parámetros relacionados que controlan el suavizado del mapa de disparidad, es recomendable que $P2$ $>$ $P1$.
    \item \textbf{Relación de unicidad:} aplica un margen de porcentaje el cual deberá ser superado por la mejor disparidad hallada por la función de coste respecto al segundo mejor valor, en caso contrario se elegirá como ganador al segundo mejor candidato.
    \item \textbf{modo:} este controla el número de direcciones mencionadas en el paso 4.
    \item \textbf{Pre-filter cap:} su efecto es notable en el paso 1, puesto que trunca el valor de los píxeles filtrados en el rango $[-preFilterCap, preFilterCap]$. 
\end{itemize}
\subsection{Post-procesamiento}
\begin{enumerate}
    \item Al mapa de disparidad obtenido se le aplica un filtro basado en mínimos cuadrados ponderados (WLS), el cual es una generalización de los mínimos cuadrados ordinarios, dicho filtro ayuda a refinar los resultados cuando existen oclusiones y areas uniformes de las cuales no es sencillo distinguir la diferencia entre pixeles.
    \item Se redimensiona al mapa de disparidad mejorado con un filtro Laplaciano, el cual consiste en inyectar las columnas y filas pares en 0 y luego aplicar la convolución con el kernel Laplaciano, realzando así los detalles de alta frecuencia y recuperando el tamaño original.
\end{enumerate}
En esta etapa es posible modificar dos parámetros, lambda que se encarga de definir el monto de regularización durante el filtrado, haciendo que valores grandes obliguen a que los bordes del mapa de disparidad filtrados se adhieran más a los bordes de la imagen de origen. El otro sería el valor de sigma que define cuán sensible es el proceso de filtrado a los bordes de la imagen de origen. Los valores grandes pueden provocar fugas de disparidad a través de bordes de bajo contraste. Los valores pequeños pueden hacer que el filtro sea demasiado sensible al ruido y las texturas de la imagen de origen. Los valores de sigma típicos oscilan entre 0,8 y 2,0.
\subsection{Reconstrucción 3D}
A partir del mapa de disparidad de un solo canal y la matriz Q, tal que:
\begin{equation}
    Q = \begin{bmatrix}
            1 & 0 & 0 & -c_{x}\\
            0 & 1 & 0 & -c_{y}\\
            0 & 0 & 0 & f\\
            0 & 0 & -\frac{1}{T_{x}} & \frac{(c_{x} - c_{x\prime})}{T_{x}}
        \end{bmatrix}
\end{equation}
Donde $c_{x}$ y $c_{y}$ son el centro óptico de un canal y $c_{x\prime}$ es la coordenada en x del otro centro óptico, $f$ es la distancia focal y $T_{x}$ se extrae de la matriz de traslación, la importancia de esta matriz se encuentra en la siguiente expresión
\begin{equation}
    Q\begin{bmatrix}
            x\\
            y\\
            d\\
            1
        \end{bmatrix} = \begin{bmatrix}
            X\\
            Y\\
            Z\\
            W
        \end{bmatrix}
\end{equation}
Y es que dicha matriz permite convertir los mapas de disparidad en mapas de profundidad donde cada punto se encuentra en coordenadas homogéneas o simplemente hallar las coordenadas de ciertos puntos respecto al sistema estéreo.
\section{Selección de técnica de reconocimiento}\label{tecnica_recog}
Se evaluaron dos estrategias una enfocada en segmentación semántica con la arquitectura de red de DeepLab V3+ y otra basada en detección de bounding boxes (YOLO V3), a continuación se listan las diferencias más relevantes entre ambas aproximaciones:
\begin{table}[H]
\renewcommand{\arraystretch}{1.5}
\centering
\caption{Contraste entre YOLO V3 y DeepLab V3+}
\label{deeplab_vs_yolo}
\begin{tabular}{|c|c|}
\hline
DeepLab V3+ & YoloV3 \\ \hline
\multicolumn{1}{|p{7cm}|}{La salida es una imagen donde se etiqueta cada píxel asignándole una clase.} & \multicolumn{1}{|p{7cm}|}{la salida es una matriz con las coordenadas de los bounding box, las clases de los objetos detectados y el grado de confidencia, que no es mas que el producto entre la probabilidad de que el objeto pertenezca a la clase y el IoU.} \\ \hline
\multicolumn{1}{|p{7cm}|}{Su arquitectura se sustenta en una red encoder-decoder} & \multicolumn{1}{|p{7cm}|}{Su arquitectura se fundamenta en una CNN del tipo ResNet} \\ \hline
\multicolumn{1}{|p{7cm}|}{Su métrica de evaluación del modelo es la mIoU, que es la IoU promedio} & \multicolumn{1}{|p{7cm}|}{Su métrica es el mAP (Mean Average Precision)} \\ \hline
\multicolumn{1}{|p{7cm}|}{Se enfoca más en la precisión que en la velocidad por lo que su tiempo de inferencia es mayor} & \multicolumn{1}{|p{7cm}|}{Alcanza un valor de precisión menor, sin embargo; su tiempo de inferencia es menor} \\ \hline
\multicolumn{1}{|p{7cm}|}{Debido a que posee un bloque ASPP es capaz de comprender de mejor forma la información contextual de objetos a diferentes escalas} & \multicolumn{1}{|p{7cm}|}{Realiza las predicciones a 3 diferentes escalas, extrayendo las features por cada escala, similar a las redes FPN \cite{FPN} }\\ \hline
\end{tabular}
\end{table}
Al ser el objetivo final el posicionamiento de objetos junto al algoritmo estéreo mencionado en la sección previa, es menester tomar en cuenta el como se realizaría dicha acción en los dos casos:
\begin{itemize}
    \item \textbf{DeepLab V3+:} con la segmentación de la imagen de entrada se buscarían las proyecciones 3D de todos los píxeles de disparidad que pertenezcan a la clase en cuestión y posteriormente se calcula el promedio en cada coordenada, de modo que el resultado sería la distancia media con el objeto.
    \item \textbf{YOLO V3:} su cálculo es un tanto más complejo debido a que los bounding boxes podrían contener píxeles que no correspondan con la clase dada, por lo que se ideó una forma que requiere un procesamiento extra, pero logra un resultado similar al de DeepLab V3+, esta consiste en que a cada región enmarcada por los bounding boxes en la imagen de entrada, se le aplique un filtro gaussiano que suavice los bordes, para luego aplicar el método de umbralización de Otsu, el cual asignara como 0 a los valores de píxeles que estén por debajo de cierto umbral (el fondo del objeto detectado) y 1 al objeto en sí mismo, de modo que a los que tengan un valor de uno se les halle sus coordenadas en el plano de imagen y finalmente se proyecten al plano 3D por medio de la matriz Q y el mapa de disparidad obtenido, entonces al igual que con la segmentación se halla el promedio.
\end{itemize}
Tomando en cuenta las características mencionadas y las cualidades del hardware empleado, se eligió la red YOLO V3, principalmente porque esta posee una arquitectura que requiere una menor cantidad de iteraciones para el entrenamiento, su velocidad es mayor, lo que permite comprobar su tiempo de inferencia en local además de en Google Colab, cuenta con una versión para dispositivos móviles que es más rápida aun, a costa de la precisión y que a pesar de que requiere un par de pasos extra para posicionar un objeto sigue siendo más veloz que una red como DeepLab V3+.
\\
\\
Adicionalmente antes de implementar YOLO V3 se experimentó con la herramienta Tensorflow API model zoo que alberga una variedad de redes neuronales y entre ellas se eligió una red similar a YOLO de nombre SSD (Single shot detector), no obstante se observó que a pesar lo buena que pueda ser la herramienta posee el problema de que para usarse depende del repositorio alojado en ``https://github.com/tensorflow/models.git'' y de la librería protobuf ambos archivos son excesivamente pesados para un paquete, puesto que esta API (Aplication Program Interface) está diseñada para entornos virtuales con mayores capacidades como pueden ser servidores dedicados, por lo que se desechó esta posibilidad para aligerar el peso del paquete y facilitar el uso del mismo.
\section{Implementación de YOLO V3}
\subsection{Arquitectura}
Se agregaron dos tipos su versión por defecto (YOLO V3) y su versión diseñada para dispositivos móviles (YOLO V3 Tiny), la primera consta de tres partes, una para la extracción de características fundamentada en la arquitectura Darknet53 \cite{yolov3} que utiliza bloques residuales para formar una ResNet, la segunda parte que se encarga de crear las tres ramas para la detección de objetos a 3 escalas diferentes y la tercera que es similar para ambas arquitecturas y es la que entrega la matriz de predicciones. YOLO V3 Tiny se diferencia en que en lugar de usar Darknet53 se sustenta en Darknet19, la otra diferencia es que en lugar de crear 3 ramas se crean solo 2 una para las predicciones de objetos de gran tamaño y otra para los objetos de tamaño mediano. 
\subsection{Entrenamiento} 
El método que permite esta función requiere que las anotaciones introducidas se encuentren en el formato adecuado, como se pudo observar en el capítulo previo esto se logra con los archivos annotations\_parser.py y annotations\_helper.py. Estos separan las anotaciones en entrenamiento, validación y en caso de ser necesario se puede separar el subconjunto de evaluación. Para entrenar también se necesita la ubicación del archivo que contenga una lista con las clases que detectara la red. Entre los hiperpárametros más importantes que el usuario puede modificar se encuentran:
\begin{itemize}
    \item \textbf{Warmup\_epochs:} este nace del hecho de que al comienzo del entrenamiento todos los hiperpárametros son típicamente aleatorios y suelen encontrarse lejos de su valor final, por esto al principio se usa una tasa de aprendizaje cercana a 0 que va creciendo linealmente bajo la expresión \ref{warmup_lineal}
    \begin{equation}
        lr = \frac{step\_n \cdot lr\_init}{warmup\_steps}\label{warmup_lineal}
    \end{equation}
    donde $lr$ es la tasa de entrenamiento, $step\_n$ es el número de iteración el cual comienza en 1 y sube con cada iteración, lr\_init es un hiperpárametro que también controla el usuario y corresponde con la tasa de entrenamiento inicial y warmup\_steps esta dado por la expresión \ref{warmup_steps}
    \begin{equation}
        warmup\_steps = \frac{warmup\_epochs \cdot dataset\_size}{batch\_size} \label{warmup_steps}
    \end{equation}
    En la expresión \ref{warmup_steps} se observa donde se involucra warmup\_epochs, dataset\_size es el tamaño del conjunto de datos total y batch\_size es otro hiperpárametro que coincide con la cantidad de imágenes que se le introducen a la red en una iteración. 
    \\
    Cuando el número de la iteración supera el valor de warmup\_steps (los parámetros y métricas comienzan a converger) la función que controla la tasa de aprendizaje cambia a:
    \begin{equation}
        lr = lr\_end + \frac{lr\_init - lr\_end}{2}\cdot \left(1 + \cos\left(\frac{(step\_n - warmup\_steps)\pi}{dataset\_size - warmup\_steps}\right) \right) \label{lr_cos}
    \end{equation}
    lr\_end es la tasa final que siempre debe ser más pequeña que la inicial, cuando la ecuación \ref{lr_cos} se activa, el $lr$ ira decayendo lentamente desde lr\_init  hasta lr\_end hasta que se acabe el entrenamiento.
    \item \textbf{epochs:} para entender su valor, es menester comprender el ciclo de la rutina de entrenamiento, la cual consiste en:
    \begin{enumerate}
        \item Introducir a la red un lote de imágenes de tamaño igual al batch\_size.
        \item Predecir o inferir la matriz de salida.
        \item Calcular el error con la función de coste el cual está compuesto por 3 errores distintos:
        \begin{enumerate}
            \item El error de clasificación o error de probabilidad
            \item El error de confianza que representa la existencia de un objeto en un bounding box.
            \item El error dado por el IoU, en este caso se utilizó la métrica GIoU que es una forma generalizada del IoU.
        \end{enumerate}
        \item Aplicar el algoritmo del descenso al gradiente con el optimizador de Adam.
        \item Ajustar la tasa de aprendizaje o learning rate con la metodología ``warmup'' previamente descrita.
        \item Repetir todos los pasos previos hasta introducir todos los datos de entrenamiento.
        \item Realizar los pasos 1, 2 y 3 para todo el conjunto de validación. 
        \item Guardar los pesos y parámetros del modelo.
    \end{enumerate}
    Las instrucciones mencionadas comprenden 1 epoch, por lo que para entrenar una red estos pasos se deben ejecutar una cantidad $n$ de epochs o épocas.
\end{itemize}
\subsection{Recuperación de pesos}
Dado que los pesos de una red conservan el entrenamiento por el que paso el modelo, se crearon dos formas para tomar los pesos de un archivo con un mismo método, existe la opción de tomarlos de un archivo conocido como ``checkpoint'' que es generado por la red al momento del entrenamiento en cada época o incluso utilizar los pesos originales de la red YOLO V3 entrenada con el conjunto de datos COCO el cual cuenta con 80 clases diferentes a detectar, crear esta opción fue necesario debido a que los pesos originales poseen un formato .weights el cual no es compatible con los métodos propios de Tensorflow para su recuperación.
\subsection{Inferencia}
Como se mencionó en el capítulo previo existen 4 formas de inferencia compatibles con cualquier modelo de detección en el archivo detection\_mode.py, sin embargo el algoritmo que realiza esta tarea en líneas generales sigue los procesos de la Figura \ref{inference_diagram}, entonces la inferencia consiste en 4 etapas, el preprocesamiento que reduce la imagen a unas dimensiones compatibles con la arquitectura de la red, la predicción que entrega una matriz de dimensiones $n$ $x$ $6$, el post-procesamiento que redimensiona los bounding boxes y la imagen al tamaño original, además de aplicar el algoritmo NMS y por último se dibujan en la imagen los bounding boxes con la clase de cada objeto y el grado de confidencia.
\begin{figure}[H]
    \centering
    \includegraphics[scale=0.6]{Recursos/inference_cycle.png}
    \caption{Ciclo de inferencia}
    \label{inference_diagram}
\end{figure}
\subsection{Evaluación}
Se utilizó la métrica mAP (Mean Average Precision) para evaluar la precisión de la red, esta se fundamenta en la IoU y los siguientes conceptos:
\begin{itemize}
    \item \textbf{Precisión:} mide cuán preciso son las inferencias y se calcula con la siguiente ecuación:
    \begin{equation}
        prec = \frac{TP}{TP + FP} = \frac{TP}{Casos\; positivos}
    \end{equation}
    donde TP (True Positive) es la cantidad de predicciones cuyo resultado coincide con el ground truth y la clase verdadera, FP (False Positive) son las predicciones cuyo resultado tiene una probabilidad mayor a un umbral definido por el usuario, pero que no coincide con el ground truth y la clase.
    \item \textbf{Recall:} mide que tan bueno es el detector para hallar todos los positivos, se mide con la siguiente expresion:
    \begin{equation}
        rec = \frac{TP}{TP + FN} = \frac{TP}{Casos Totales}
    \end{equation}
    FN (False Negative): son las predicciones que no superan el umbral definido, pero que aun así coinciden con el resultado real.
\end{itemize}
El valor de AP (Average Precision) esta definido como:
\begin{equation}
    AP = \int_{0}^{1} p(r)dr
\end{equation}
donde $p(r)$ es el área bajo la curva precisión vs. recall que se forma con todos los datos de evaluación, AP es calculado para cada clase por separado, de tal modo que mAP es el promedio formado por todas las clases del modelo. 
\section{Posicionamiento de objetos}
Al ser este el objetivo final del paquete la clase VisionSystem abarca la rutina de cálculo de disparidad más detección para imágenes, streaming y video. Asumiendo que previamente se realizó la calibración de las cámaras, se obtuvieron mapas que rectifican las imágenes izquierda y derecha del sistema estéreo y se hallaron los parametros adecuados para el algoritmo SGBM. Adicionalmente requiere que el modelo de AA haya sido entrenado previamente para obtener un resultado satisfactorio.
\\
\\
Con todos estos elementos en la Figura \ref{vision_system_cycle} se puede observar como la imagen de la izquierda es sobreescrita para obtener el resultado final, de igual forma el filtro WLS (mínimos cuadrados ponderados) que corresponde con el post-procesamiento estéreo utiliza como referencia el mapa de disparidad izquierdo, por lo que el mapa refinado corresponde con la imagen de la izquierda y el cálculo de distancia sigue la misma metodología descrita en la sección \ref{tecnica_recog}.
\begin{figure}[H]
    \centering
    \includegraphics[scale=0.6]{Recursos/vision_system_cycle.png}
    \caption{Diagrama para el posicionamiento con visión estéreo y YOLO V3}
    \label{vision_system_cycle}
\end{figure}

%==================================================================
\chapter{PRUEBAS EXPERIMENTALES Y ANÁLISIS DE RESULTADOS}\label{CAP:resultados}
%\markboth{Tu Segundo Capítulo}{Tu Segundo Capítulo}%
En este capítulo se expondrán las pruebas experimentales de las principales funcionalidades del paquete, los resultados obtenidos y por supuesto las observaciones de cada prueba de forma separada. Además de un análisis global del comportamiento del paquete para lograr el objetivo señalado en el título de este proyecto.
\section{Calibración y rectificación de las cámaras}
Las cámaras IMX-219 poseen un FOV de 160$^o$ por lo que en la clase StandardStereo se fijó la variable fish\_eye como verdadera, lo que permitió alcanzar un error RMS inferior a 1 al calibrar y en consecuencia las líneas epipolares en las imágenes rectificadas lograron ser paralelas.
\subsection{Calibración}
\begin{enumerate}
    \item Se ejecutó un servidor local con el Stereopi V2, configurando el sensor de la cámara en 6560 x 2464 píxeles, sin embargo las imágenes reales poseen dimensiones de 3280 x 2464 píxeles, debido a que cada foto tomada por el módulo posee una disposición como la de la Figura \ref{side-by-side} donde en una sola imagen se encuentran ambas escenas, por lo que cada imagen fue recortada por la mitad con la clase SplitPair y reducida a dimensiones de 640 x 720 con el fin de obtener tiempos de procesamiento aceptables.
    \begin{figure}[H]
    \centering
    \includegraphics[scale=0.5]{Recursos/side_by_side.png}
    \caption{Disposición de fotos tomadas por StereoPi V2}
    \label{side-by-side}
    \end{figure}
    \item Se ejecutaron los pasos de la sección \ref{calibration_section} del capítulo II con 42 imágenes de calibración y con 20, en la tabla \ref{calibration_results} se pueden observar los resultados obtenidos.
    \begin{table}[H]
    \centering
    \caption{Error RMS de calibración}
    \label{calibration_results}
    \begin{tabular}{c|ccc|}
    \cline{2-4}
    \multicolumn{1}{l|}{}                & \multicolumn{3}{c|}{Error}                                                            \\ \hline
    \multicolumn{1}{|c|}{$N^o$ imágenes} & \multicolumn{1}{c|}{cámara izquierda} & \multicolumn{1}{c|}{cámara derecha} & estéreo \\ \hline
    \multicolumn{1}{|c|}{42}             & \multicolumn{1}{c|}{0.31}             & \multicolumn{1}{c|}{0.19}           & 0.28    \\ \hline
    \multicolumn{1}{|c|}{20}             & \multicolumn{1}{c|}{0.20}             & \multicolumn{1}{c|}{0.19}           & 0.21    \\ \hline
    \end{tabular}
    \end{table}
\end{enumerate}
\subsection{Rectificación}
En la Figura \ref{rectification_result} se puede apreciar el efecto de la rectificación.
\begin{figure}[H]
    \centering
    \includegraphics[scale=0.6]{Recursos/rectification_result.jpg}
    \caption{Efecto de la rectificación}
    \label{rectification_result}
\end{figure}
Para verificar la calidad de la rectificación en los dos casos expuestos de la tabla \ref{calibration_results} se dibujaron las líneas epipolares con el método find\_epilines, el cual pertenece a la clase StandardStereoBuilder obteniendo así las Figuras \ref{epilines_20} y \ref{epilines_42}
\begin{figure}[H]
    \centering
    \includegraphics[scale=0.4]{Recursos/epilines_20_calibration_images.jpg}
    \caption{Líneas epipolares para 20 imágenes de calibración}
    \label{epilines_20}
\end{figure}
\begin{figure}[H]
    \centering
    \includegraphics[scale=0.3]{Recursos/epilines_42_calibration_images.jpg}
    \caption{Líneas epipolares para 42 imágenes de calibración}
    \label{epilines_42}
\end{figure}
\subsection{Análisis de resultados}
Como se puede observar en la tabla \ref{calibration_results} el número propuesto por Zhengyou Zhang \cite{Zhang2000} es acertado, puesto que la calibración es un método iterativo, demasiadas imágenes pueden llegar a ser contraproducente, debido a que el error termina divergiendo y en lugar de reducirse comienza a aumentar, esto se termina de comprobar en las Figuras \ref{epilines_20} y \ref{epilines_42}, donde a pesar de que en ambas imágenes dichas líneas son paralelas precisamente la que utiliza 20 imágenes logra un mejor resultado que permite reducir verdaderamente el problema de correspondencia, cabe resaltar que a pesar de reducir las dimensiones de las imágenes de entrada, esto no afecta en el cálculo de los parámetros de las cámaras y del sistema estéreo en sí, incluso es posible ingresar otras dimensiones en el paso de rectificación en caso de ser necesario. 
\\
\\
Adicionalmente se logra apreciar que el proceso de rectificación fue satisfactorio, porque las imágenes de muestra en la Figura \ref{rectification_result} eliminan las distorsiones causadas por las cualidades de las cámaras y especialmente el efecto de distorsión radial del lente.
\section{Cálculo de disparidad}
Se captaron 9 imágenes en un entorno cerrado y 2 en un entorno abierto (redimensionadas a 640 x 720 píxeles) bajo diferentes perspectivas para observar el comportamiento de los mapas de disparidad ante los ajustes de los parámetros del algoritmo estéreo SGBM, dichas imágenes se pueden encontrar en el repositorio de github del paquete.
\subsection{Ajuste de parámetros}
Se creó un script para el ajuste de los parámetros del algoritmo estéreo, dicho archivo se encuentra en el repositorio https://github.com/corvus96/PyTwoVision dentro de la carpeta tutorials. Se seleccionaron 3 conjuntos de ajustes los cuales se pueden observar en la tabla \ref{tune_parameters}
\begin{table}[H]
\centering
\caption{Ajuste de valores algoritmo de correspondencia}
\label{tune_parameters}
\begin{tabular}{|c|ccc|}
\hline
                             & \multicolumn{3}{c|}{Configuraciones}                                                    \\ \hline
Parámetros                   & \multicolumn{1}{c|}{A}             & \multicolumn{1}{c|}{B}             & M             \\ \hline
Disparidad mínima            & \multicolumn{1}{c|}{-32}           & \multicolumn{1}{c|}{-33}           & -32           \\ \hline
Número de disparidades       & \multicolumn{1}{c|}{32}            & \multicolumn{1}{c|}{32}            & 32            \\ \hline
Tamaño de bloque             & \multicolumn{1}{c|}{3}             & \multicolumn{1}{c|}{3}             & 3             \\ \hline
Disp12MaxDiff                & \multicolumn{1}{c|}{-38}           & \multicolumn{1}{c|}{-38}            & -38            \\ \hline
Rango de motas               & \multicolumn{1}{c|}{1}             & \multicolumn{1}{c|}{9}             & 5             \\ \hline
Tamaño de la ventana moteada & \multicolumn{1}{c|}{114}           & \multicolumn{1}{c|}{120}           & 117           \\ \hline
P1                           & \multicolumn{1}{c|}{89}           & \multicolumn{1}{c|}{125}            & 107           \\ \hline
P2                           & \multicolumn{1}{c|}{487}          & \multicolumn{1}{c|}{932}          & 710          \\ \hline
Relación de unicidad         & \multicolumn{1}{c|}{3}             & \multicolumn{1}{c|}{3}             & 3             \\ \hline
modo                         & \multicolumn{1}{c|}{8 direcciones} & \multicolumn{1}{c|}{8 direcciones} & 8 direcciones \\ \hline
Pre-filter cap               & \multicolumn{1}{c|}{14}            & \multicolumn{1}{c|}{57}            & 36            \\ \hline
Lambda                       & \multicolumn{1}{c|}{8214}         & \multicolumn{1}{c|}{19132}         & 13673         \\ \hline
Sigma                        & \multicolumn{1}{c|}{1.589}         & \multicolumn{1}{c|}{1.046}          & 1.318         \\ \hline
\end{tabular}
\end{table}
A continuación se presentarán los mapas de disparidad para las 3 configuraciones sin aplicar el paso de post-procesamiento:
\begin{figure}[H]
    \centering
    \includegraphics[scale=0.6]{Recursos/disparity_maps_without_postprocessing_results.jpg}
    \caption{Mapas de disparidad para las 3 configuraciones sin post-procesamiento}
    \label{disparity_without_postprocess}
\end{figure}
Y aplicando post-procesamiento:
\begin{figure}[H]
    \centering
    \includegraphics[scale=0.6]{Recursos/disparity_maps_with_postprocessing_results.jpg}
    \caption{Mapas de disparidad para las 3 configuraciones con post-procesamiento}
    \label{disparity_with_postprocess}
\end{figure}
\subsection{Tiempo de ejecución}
Dado que el código estéreo no utiliza procesamiento paralelo, hecho que se comprobó ejecutándolo en la plataforma de Google Colab obteniendo resultados similares a los alcanzados en el computador local, se obtuvo la velocidad de procesamiento de la etapa de correspondencia y de post-procesamiento en el computador local, al aplicar un redimensionamiento en las imágenes rectificadas, por lo que en la tabla \ref{speed_matching_results} se pueden observar los valores obtenidos.
\begin{table}[H]
\caption{Efecto del redimensionamiento en la velocidad del proceso estéreo}
\label{speed_matching_results}
\begin{tabular}{|cc|c}
\cline{1-2}
\multicolumn{2}{|c|}{Tiempo de ejecución de etapa}         &                                           \\ \hline
\multicolumn{1}{|c|}{correspondencia} & post-procesamiento & \multicolumn{1}{c|}{factor de escala} \\ \hline
\multicolumn{1}{|c|}{0.1691}          & 0.0249             & \multicolumn{1}{c|}{1}                    \\ \hline
\multicolumn{1}{|c|}{0.0374}          & 0.0034             & \multicolumn{1}{c|}{1/2}                  \\ \hline
\multicolumn{1}{|c|}{0.0090}          & 0.0022             & \multicolumn{1}{c|}{1/4}                  \\ \hline
\multicolumn{1}{|c|}{0.0024}          & 0.0017             & \multicolumn{1}{c|}{1/8}                  \\ \hline
\end{tabular}
\end{table}
Es importante recalcar que el ajuste de dimensiones aplicado, corresponde con el mencionado en el paso de preprocesamiento, por lo que el ancho y alto nuevo serán $ancho\cdot factor$ y $alto\cdot factor$ de modo que la imagen pre-procesada tendrá 1/4 de la dimensión original en el caso de ser 1/2.
\subsection{Análisis de resultados}
Al comparar las Figuras \ref{disparity_without_postprocess} y \ref{disparity_with_postprocess} se observa que efectivamente el paso de post-procesamiento mejora notablemente la calidad de los mapas de disparidad y su tiempo de ejecución es entre un 10\% a 20\% la duración del paso de correspondencia, el ajuste de parámetros es una tarea que requiere iterar constantemente hasta conseguir los mejores resultados posibles, en este caso se ajustaron primero los casos A y B, mientras que la configuración M surgió de hallar el valor promedio entre los parámetros de A y B. 
\\
\\
Por otro lado en la tabla \ref{speed_matching_results} se aprecia como claramente la reducción de las imágenes rectificadas tiene un efecto considerable en el tiempo de ejecución, sin olvidar que estos tiempos están asociados al procesador utilizado. A pesar de ello, no se debe abusar de la reducción de tamaño, porque si bien es cierto que el método usado para redimensionar logra recuperar más información que un método convencional, al recuperar el tamaño original de una imagen la cual fue reducida con un factor de escala por encima de un 1/4 pueden ocurrir dos cosas:
\begin{itemize}
    \item Si la imagen rectificada se encontraba desplazada al menos unos píxeles, es decir que existiese una banda vertical en los laterales de la imagen, la imagen final tendría las mismas bandas, pero con un mayor grosor, lo que reduciría el campo de visión.
    \item Los detalles u objetos pequeños del mapa de disparidad podrían verse eliminados.
\end{itemize}
No obstante, el límite de factor de escala dependerá del tamaño de las imágenes de entrada.
\section{Entrenamiento de YOLO V3}
Para realizar el entrenamiento se utilizó la plataforma de Google Colab y las gráficas que representan la evolución de la red en el entrenamiento fueron generadas con Tensorboard (herramienta de Tensorflow).
\subsection{Conjunto de datos}
Se seleccionó el conjunto de datos PASCAL VOC 2012 \cite{pascal-voc-2012}, el cual cuenta con 20 clases repartidas en 17125 imágenes, dicho conjunto fue dividido de forma aleatoria en datos de entrenamiento asignándole el 80\% (13700) y datos de validación con el 20\% restante (3425).
\subsection{Hiperparámetros}
\begin{itemize}
    \item warmup\_epochs = 5
    \item lr\_init = $10^{-3}$
    \item lr\_end = $10^{-6}$
    \item batch\_size = 20, a consecuencia de este tamaño de lote dado que en una iteración son introducidas 20 imágenes a la red, para completar un epoch se necesitaron 685 iteraciones.
    \item epochs = 160, sin embargo se detuvo el entrenamiento en el número 75 en vista de que el error de validación comenzaba a subir en lugar de bajar desde el epoch número 37.
\end{itemize}
Para el resto de los hiperparámetros se seleccionaron las opciones por defecto. 
\subsection{Rutina de entrenamiento/validación}
El tiempo de entrenamiento fue de 19.28 horas, no obstante el mejor resultado fue alcanzado en el epoch número 37, a las 9.50 horas, este hecho se puede observar claramente en la Figura \ref{validation_result} donde se alcanza el punto de inflexión precisamente en dicho epoch (iteración número 25354), cabe resaltar que a pesar de que el error de entrenamiento continúa descendiendo (Ver Figura \ref{training_result}) luego de dicho número de iteraciones, por norma cuando el error de validación comienza a subir la red tiende a sufrir de overfitting (sobre ajuste).
\begin{figure}[H]
    \centering
    \includegraphics[scale=0.4]{Recursos/training_loss_result.jpg}
    \caption{Errores de entrenamiento (Error vs. Número de iteración)}
    \label{training_result}
\end{figure}
\begin{figure}[H]
    \centering
    \includegraphics[scale=0.6]{Recursos/validation_loss_result.jpg}
    \caption{Errores de validación (Error vs. Número de epoch)}
    \label{validation_result}
\end{figure}
En la Figura \ref{learning_rate_result} se observa como el learning rate evoluciono con cada iteración y efectivamente este cumple la regla impuesta por el warmup\_epochs de crecer linealmente durante 5 epochs (3425 iteraciones) y comenzar a decaer luego de dicho límite.
\begin{figure}[H]
    \centering
    \includegraphics[scale=0.6]{Recursos/learning_rate_evolution_result.jpg}
    \caption{Evolución del learning rate (learning rate vs. Número de iteración)}
    \label{learning_rate_result}
\end{figure}
Es menester resaltar que el error de validación alcanzado en el epoch 37 es:
\begin{itemize}
    \item error de clasificacion o probabilidad = 7.60 \%.
    \item error de confianza = 5.95 \%.
    \item error por GIoU = 3.14 \%.
\end{itemize}
Por lo que el error total es de 16.69 \%.
\subsection{Evaluación}
En las Figuras \ref{mAP_50_result}, \ref{mAP_75_result}, \ref{mAP_100_result}, se aprecia el cálculo de la métrica mAP para cada clase y el total, como era de esperar a medida que el umbral se eleva el resultado final disminuye.
\begin{figure}[H]
     \centering
     \begin{subfigure}[b]{0.4\textwidth}
        \centering
        \includegraphics[scale=0.15]{Recursos/mAP50_result.jpg}
        \caption{$mAP_{50}$}
        \label{mAP_50_result}
     \end{subfigure}
     \hfill
     \begin{subfigure}[b]{0.4\textwidth}
         \centering
        \includegraphics[scale=0.15]{Recursos/mAP75_result.jpg}
        \caption{$mAP_{75}$}
        \label{mAP_75_result}
     \end{subfigure}
     \hfill
     \begin{subfigure}[b]{0.4\textwidth}
         \centering
        \includegraphics[scale=0.15]{Recursos/mAP100_result.jpg}
        \caption{$mAP_{100}$}
        \label{mAP_100_result}
     \end{subfigure}
\caption{Mean Average precision (mAP) para distintos umbrales}
\label{mAP_results}
\end{figure}
\subsection{Tiempo de ejecución}
Se evaluó dicho tiempo tanto para el computador local como en el servidor de Google Colab, obteniendo aproximadamente 1.8 frames por segundo (555.56 ms) en el computador local y 12.36 FPS (80.91 ms) en la plataforma de Colab cuando se utiliza la GPU.
\subsection{Análisis de resultados}
Es notable que aunque el error de validación se encuentre por encima del valor recomendable que debería ser por debajo del 1\%, aun así el modelo es capaz de obtener resultados aceptables para inferir en un ambiente cerrado con buena iluminación, como se puede ver en la Figura \ref{training_result} los errores de entrenamiento se mantienen siempre por debajo de los de validación, esto se debe a que cuando se realiza la validación, la red no está ajustando los pesos, ya que solo está comprobando el entrenamiento al introducir nuevos datos, por este motivo se trata de evitar el overfitting dividiendo el conjunto de datos, con el fin de que la red pueda generalizar de mejor manera.
\\
\\
Además en las Figuras \ref{training_result} y \ref{validation_result} se observa ruido (el error es infinito) después del epoch 60 esto se debe a que en ese punto se desconectó la máquina virtual de Colab y al volver a conectar la red continuó el entrenamiento con los mismos pesos, pero aplicando la estrategia de warmup partiendo con el mínimo learning rate posible, esto en sí mismo es un error debido a que cuando se tienen pesos ya entrenados la etapa de warmup de 5 epochs en este caso es innecesaria, lo correcto seria entrenar con un warmup de 0.
\\
\\
Otro motivo por el que puede que la red no sea capaz de mejorar sus resultados, es debido a que varios autores recomiendan al menos 1000 imágenes por clase, para conseguir una buena generalización, sin embargo; para esto la red debería tener al menos 20000 imágenes, una solución a esto podría ser tomar pesos ya entrenados y aplicar transferencia de aprendizaje. En cuanto a la velocidad de la red el tiempo de predicción se reduce significativamente con el uso de una GPU dedicada como la de Colab.
\section{Posicionamiento}
Para estas pruebas se emplearon las configuraciones A, B y M de los mapas de correspondencia y se cargaron los pesos del conjunto de datos COCO con el método restore\_weights de la clase Recognizer. COCO cuenta con 80 clases y más de 200 mil imágenes, su mayor variedad de objetos y mejor precisión fue la razón de su elección para comprobar la ubicación de objetos de diferentes características.
\subsection{Efecto de umbralización de Otsu}
Uno de los pilares de la metodología de posicionamiento es la segmentación por umbralización de Otsu y a pesar de que no es perfecta, dadas las condiciones impuestas, es posible obtener resultados favorables, tal es el caso de la Figura \ref{otsu_bike}, que aunque tiene una línea de píxeles que no corresponden con el objeto detectado, la mayoría de ellos si pertenecen a dicho objeto y dado que la posición es la media de las coordenadas de todos esos píxeles, la distancia tiende a ser la del objeto en sí mismo y no la del fondo de su bounding box. 
\\
\\
No obstante, hay casos en los que no siempre funciona adecuadamente como se puede ver en la Figura \ref{otsu_inverso} que en lugar de segmentar la botella de plástico sucede lo contrario esto es debido a que este método en su versión inversa favorece los píxeles más oscuros, por este motivo para tratar con ese tipo de objetos se agregó una variable booleana que permite alternar entre la umbralización inversa y normal, la segunda favorece a píxeles más cercanos al blanco y es bastante preciso en segmentar botellas transparentes como es el caso de la Figura \ref{otsu_normal}.
\begin{figure}[H]
     \centering
     \begin{subfigure}[b]{0.4\textwidth}
        \centering
        \includegraphics[scale=0.35]{Recursos/otsu_bike.jpg}
        \caption{Umbralización en una bicicleta}
        \label{otsu_bike}
     \end{subfigure}
     \hfill
     \begin{subfigure}[b]{0.4\textwidth}
        \centering
        \includegraphics[scale=0.35]{Recursos/otsu_position.jpg}
        \caption{Umbralización en una flor}
        \label{otsu_position}
     \end{subfigure}
     \hfill
     \begin{subfigure}[b]{0.4\textwidth}
        \centering
        \includegraphics[scale=0.35]{Recursos/umbral_inverso.jpg}
        \caption{Umbralización inversa}
        \label{otsu_inverso}
     \end{subfigure}
     \hfill
     \begin{subfigure}[b]{0.4\textwidth}
         \centering
        \includegraphics[scale=0.31]{Recursos/umbral_normal.jpg}
        \caption{Umbralización normal}
        \label{otsu_normal}
     \end{subfigure}
\caption{Segmentación binaria de Otsu}
\label{OTSU_segmentation}
\end{figure}
Para las pruebas realizadas a continuación se eligió la versión inversa debido a que es más probable que los objetos detectados se encuentren superpuestos a su fondo y por ende sus píxeles sean más oscuros.
\subsection{Posicion real vs. configuraciones A, B y M}
Para convertir los mapas de disparidad en distancia se usó la siguiente matriz Q:
\begin{align}
    Q = \begin{bmatrix}
            1 & 0 & 0 & -320\\
            0 & 1 & 0 & -360\\
            0 & 0 & 0 & 309\\
            0 & 0 & -1.5385 & 0
            \end{bmatrix}
\end{align}
Donde el valor de la distancia focal fue extraído de la matriz de la cámara y la línea base fue medida directamente en el sistema.
\\
\\
Se tomó como elemento de prueba la flor de la imagen \ref{otsu_position}, la cual se encuentra a una distancia de $0.35 \pm 0.01$ m del sistema estéreo y se generaron las Figuras \ref{pos_conf_A}, \ref{pos_conf_B} y \ref{pos_conf_M}, cuyas coordenadas se encuentran en la escala de metros.
\begin{figure}[H]
    \centering
    \includegraphics[scale=0.5]{Recursos/position_configuration_A.jpg}
    \caption{Mapa de disparidad y coordenadas con configuración A}
    \label{pos_conf_A}
\end{figure}
El error absoluto del objeto para la configuración A es de 0.18
\begin{figure}[H]
    \centering
    \includegraphics[scale=0.5]{Recursos/position_configuration_B.jpg}
    \caption{Mapa de disparidad y coordenadas con configuración B}
    \label{pos_conf_B}
\end{figure}
El error absoluto del objeto para la configuración B es de 0.17
\begin{figure}[H]
    \centering
    \includegraphics[scale=0.5]{Recursos/position_configuration_M.jpg}
    \caption{Mapa de disparidad y coordenadas con configuración M}
    \label{pos_conf_M}
\end{figure}
El error absoluto del objeto para la configuración M es de 0.19
\\
\\ Por lo que el error absoluto promedio será de 0.18 m.
\section{Análisis de resultados}
Debido a que en la metodología seleccionada los píxeles son elegidos a partir de la imagen de entrada por el algoritmo de segmentación binaria, estos son iguales para todas las configuraciones, por lo que la posición varía conforme al mapa de disparidad. 
\\
En los 3 casos presentados se puede observar que sus mapas de disparidad son bastante similares y aunque el error es de 18 cm para objetos cercanos, a medida que un objeto se aleja superando 1 metro, la distancia mostrada no se encuentra bien representada, sin embargo estas tres configuraciones mantienen la relación de orden entre objetos de forma correcta, es decir; si un objeto se encuentra antes que otro este siempre tendrá una menor distancia en la coordenada Z.
\\
\\
El problema antes mencionado radica en que como se puede ver en la Figura \ref{disparity_with_postprocess} estos mapas de disparidad no tienen un paso gradual entre el rojo y el azul, además de que los pixeles más cercanos deberían reflejar un rojo más intenso (el rojo indica cercanía) esto se debe a que cuando se ajustaron los parámetros se buscó obtener un resultado que al menos respetara el orden entre objetos, por lo que para conseguir mejores resultados es menester tomar en cuenta un ajuste más preciso y que la imagen utilizada en dicho ajuste posea una mayor distinción de la profundidad, lo que se traduce en una mayor cantidad de objetos superpuestos y de ser posible una perspectiva que permita percibir mejor toda la profundidad de la escena. En cuanto a la velocidad sé probo el algoritmo con video y este alcanzo en promedio 1.7 FPS en la máquina local, lo que implica que la rapidez se encuentra limitada únicamente por el modelo de reconocimiento, ya que el efecto del algoritmo estéreo es mínimo en comparación.

%==================================================================
\chapter{CONCLUSIONES}\label{CAP:conclu}
%\markboth{Tu Segundo Capítulo}{Tu Segundo Capítulo}%
Encontrar la posición de objetos a través de un sistema estéreo que utiliza la disparidad como regla de medición y la detección de objetos como forma de reconocimiento, es un proceso que puede ser resumido en: identificar los parámetros de las cámaras, pre-procesar las imágenes captadas, hallar el mapa de disparidad, predecir los bounding boxes, segmentar el área de dichos recuadros y finalmente calcular las coordenadas de los píxeles segmentados mediante triangulación fundamentada en los mapas de disparidad y las cualidades físicas del sistema.
\\
Aunque el proceso parezca complejo, este se simplifica con el paquete Py2vision, de tal forma que el usuario solo tenga que concentrar sus esfuerzos en ajustar los parámetros de los mapas de disparidad a través de una interfaz y proporcionar el conjunto de datos necesarios para el entrenamiento de una red junto a sus anotaciones en un formato compatible. De igual forma el paquete está diseñado para que pueda ser implementado dicho sistema en pocas líneas de código o probando diferentes configuraciones de red, con un algoritmo de correspondencia distinto a SGBM. Al mismo tiempo tiene la capacidad de hallar un mapa de profundidad que pueda ser usado para el modelado 3D, hallar únicamente el mapa de disparidad o calibrar una sola cámara, es decir; el paquete otorga flexibilidad de implementación para cualquiera de sus principales cualidades.
\\
\\
A su vez cuenta con un conjunto de lecciones para aprender a utilizar sus funcionalidades y una documentación completa de cada clase, método y función. No obstante, lo más importante es que elimina la complejidad de implementar un sistema estéreo o de reconocimiento, el cual puede ser utilizado en robótica y aplicaciones domésticas de visión por computador.
\\ 
En definitiva los sistemas aplicables con py2vision tendrán un grado de precisión que dependerá del conjunto de datos de entrenamiento, así como de la selección de sus hiperparámetros y la calidad del mapa de disparidad obtenido. Mientras que su velocidad, está relacionada con la potencia gráfica de la GPU, ya que su límite, es el propio del modelo utilizado en el reconocimiento, en este caso fue YOLO V3.

%==================================================================
\chapter{RECOMENDACIONES}\label{CAP:recomendaciones}
%\markboth{Tu Segundo Capítulo}{Tu Segundo Capítulo}%
Es importante que en la escuela de ingeniería eléctrica de la UCV se profundice en el uso de técnicas de visión para el posicionamiento y estimación de distancias, puesto que a pesar de que sus resultados no son tan precisos como los de un sensor LIDAR, estas pueden generar implementaciones en campos tan extensos como lo son el de la robótica o el de visión por computador, que logran ser económicamente más rentables y en conjunto con otro tipo de sensores mediante fusión sensorial pueden alcanzar resultados notables en dichos campos.
\\
\\
Por este motivo se recomienda colaborar en el proyecto de código abierto py2vision para que se convierta en una herramienta de visión estéreo que cuente con una amplia variedad de técnicas de medición y diversidad de estrategias de reconocimiento.

\appendix
%==================================================================
\chapter{MANUAL DE USO DEL PAQUETE}\label{CAP:anexo0}
%\markboth{Tu anexo}{Tu anexo}%
En la Figura \ref{doc_contents} se puede apreciar la tabla de contenidos de la documentación generada para el paquete.
En el siguiente enlace https://corvus96.github.io/PyTwoVision/html/index.html se puede encontrar la documentación completa del mismo.
\begin{figure}[H]
    \centering
    \includegraphics[scale=0.5]{Recursos/doc_contents.jpg}
    \caption{Tabla de contenidos de la documentación del paquete}
    \label{doc_contents}
\end{figure}
En el enlace https://github.com/corvus96/PyTwoVision/tree/master/tutorials se encuentran los tutoriales desarrollados, los cuales pueden ser ejecutados en Jupyter notebooks o en Google Colab. 
\begin{figure}[H]
    \centering
    \includegraphics[scale=0.6]{Recursos/lesson_example.jpg}
    \caption{Extracto del tutorial de implementación de YOLO}
    \label{fig:my_label}
\end{figure}%

%==================================================================
\chapter{HERRAMIENTAS ADICIONALES DEL PAQUETE}\label{CAP:anexo1}
%\markboth{Tu anexo}{Tu anexo}%
\begin{figure}[H]
    \centering
    \includegraphics[scale=0.5]{Recursos/tuner_interface.jpg}
    \caption{Interfaz de ajuste de parámetros para SGBM}
    \label{tuner}
\end{figure}%

%\backmatter
%==================================================================
%\chapter{TÍTULO DEL ANEXO}\label{CAP:anexo2}
%\markboth{Tu anexo}{Tu anexo}%
%\input{apendice2.tex}%



%\backmatter

%==================================================================
\newpage
%\markboth{Referencias}{Referencias}%
%\addcontentsline{toc}{chapter}{Referencias}%

% References here (outcomment the appropriate case)
% CASE 1: BiBTeX used to constantly update the references (while the paper is being written).
%\bibliographystyle{IEEEtranSN}%{IEEEtranS}%%% outcomment this and next line in Case 1 siam
\printbibliography[title={REFERENCIAS}]
\newpage
%\bibliography{biblioteca} % if more than one, comma separated and without extension bib


% CASE 2: BiBTeX used to generate EIETdeG.bbl (to be further fine tuned)
%\input{EIETdeG.bbl} % outcomment this line in Case


%==================================================================
%\markboth{index}{index}%
%\addcontentsline{toc}{chapter}{Índice alfabético}%
\printindex%

\end{document}